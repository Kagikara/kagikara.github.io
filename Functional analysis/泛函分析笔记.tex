\documentclass[lang=cn,10pt]{elegantbook}

\title{泛函分析笔记}


\extrainfo{咕咕嘎嘎}

\setcounter{tocdepth}{3}
\usepackage{pdfpages}
\cover{mark-timberlake-e15DPyi2wO8-unsplash (1).pdf}
\usepackage{array}
\newcommand{\ccr}[1]{\makecell{{\color{#1}\rule{1cm}{1cm}}}}

\usepackage{amssymb}
\usepackage{braket}
% 修改标题页的橙色带
\definecolor{customcolor}{RGB}{26, 47, 107}
\colorlet{coverlinecolor}{customcolor}

\begin{document}
	
	\definecolor{structurecolor}{RGB}{26, 47, 107}
	\definecolor{main}{RGB}{46, 74, 157}
	\definecolor{second}{RGB}{70, 107, 176}
	\definecolor{third}{RGB}{90, 125, 203}
	
	\maketitle
	\frontmatter
	
	\tableofcontents
	
	\mainmatter
	
	\chapter{度量空间}
	
	\begin{introduction}
		\item 度量空间及其衍生空间
		\item 度量空间中的拓扑性质
	\end{introduction}
	
	\section{度量空间}
	
	\subsection{度量空间}
	度量空间是集合和度量函数\(d(\cdot,\cdot)\)构成的有序对\((X,d)\):
	\begin{definition}[度量空间]
		设集合\(X\)非空,若有\(d(x,y):X\times X \longmapsto \mathbb{R},\)且满足下面的条件:
		\begin{enumerate}
			\item (非负性和正定性):\(\forall x,y \in X,d(x,y)\ge0,\)且\(d(x,y)=0\)当且仅当\(x=y\);
			\item (对称性):\(d(x,y)=d(y,x);\)
			\item (三角不等式):\(d(x,z)\ge d(x,y)+d(y,z)\)
			
			满足上面条件的有序对\((X,d)\)构成度量空间,\(d\)成为\(X\)上的度量.
		\end{enumerate}
	\end{definition}
	在度量空间中,我们得以用实数从度量的角度拓展极限的含义。
	\begin{definition}[度量空间中的极限]
		设\((X,d)\)是度量空间,\(\{x_n\}\)是\(X\)中的点列,\(x_0 \in X\),若
		\[\lim_{n\to \infty}d(x_n,x_0)=0\]
		则称点列\(\{x_n\}\)按照度量\(d\)收敛于\(x_0\),记作\(x_n \to x_0\).
		
		有了度量下的极限概念,我们还可以平行定义Cauchy列:\(\forall \epsilon>0,\exists N\in \mathbb{N}_+,s.t.\forall m,n>N,\)都有
		\[d(x_m,x_n)<\epsilon,\]
		则称点列\(\{x_n\}\)是\(X\)中的Cauchy列.
	\end{definition}
	\begin{property}
		度量的性质
		\begin{enumerate}
			\item 若点列\(\{x_n\}\)收敛,则极限唯一.\textit{(三角不等式)}
			\item (度量的连续性)若\(x_n \to x_0,y_n\to y_0,\)则
			\[\lim_{n\to \infty}d(x_n,y_n)=d(x_0,y_0).\]
			\textit{(三角不等式及对称性)}
		\end{enumerate}
	\end{property}
	度量空间的完备性和可分性将在度量空间中的拓扑性质中涉及。
	\begin{example}
		不同的度量可能诱导了相同的拓扑.如在\(n\)维线性空间\(\mathbb{R}^n\)中定义度量:
		\[d_p(x,y)=(\sum_{i=1}^{n} |x_i-y_y|^p)^{1/p},p>1;\]
		\[d_\infty(x,y)=\max_{1\le i\le n}|x_i-y_i|.\]
		可知\(\lim_{n \to \infty}d_p(x_n,x_0)=0 \Leftrightarrow \lim_{n \to \infty}d_{\infty}(x_n,x_0)=0\),他们都等价于\(\{x_n\}\)按坐标收敛,即等价于\[\lim_{n \to \infty}|x_n^{(i)}-x_0^{(i)}|=0,1\le i\le n.\]
	\end{example}
	\begin{example}
		当然不同的度量也有可能诱导不同的拓扑,如在集合\(C[0,1]\)中,定义度量
		\[d_c(x,y)=\max_{0\le t \le 1}|x(t)-y(t)|;\]
		\[d_{L^1}(x,y)=\int_{0}^{1}|x(t)-y(t)|dt;\]
		考虑函数列\[g_n(x)=n^3(\frac{1}{n^2}-x)\chi_{[0,\frac{1}{n^2}]},\]
		\(g_n(x)\)在\(d_{L^1}\)下收敛于0,但是在\(d_c\)下发散至\(+\infty\).
	\end{example}
	
	\subsection{赋范线性空间}
	具有线性性的线性空间上如果再定义“距离”范数,就成为了赋范线性空间.完备的赋范线性空间称为Banach空间.
	\begin{definition}[赋范线性空间]
		记$\mathbb{K}$为实或复数域,对$\mathbb{K}$上的线性空间\(X\),若有\(||\cdot||:X\longmapsto \mathbb{R},\)且满足以下条件:
		\begin{enumerate}
			\item \(\forall x \in X, ||x|| \ge 0;\)
			\item \(\forall \alpha \in \mathbb{K},||\alpha x||=|\alpha|\cdot||x||;\)
			\item \(||x+y||\le ||x||+||y||;\)
			\item \(||x||=0 \)当且仅当\(x=0;\)
			
			则称\(||\cdot||\)为\(X\)上的范数,\((X,||\cdot||)\)成为赋范线性空间.
		\end{enumerate}
	\end{definition}
	\begin{example}
		在几乎处处相等等价类下,证明\(L^p(E)\)空间的完备性.
		\begin{proof}
			容易验证范数\(||\cdot||_p\)满足范数性质的前两条,而三角不等式由Minkovski不等式给出,于是验证了范数\(||\cdot||_p\)是\(L^p(E)\)的范数.而\(||f||_p=0\)推出\(f=0\)只有在几乎处处意义下才成立,因此等价类只能选择几乎处处相等,才能保证\((X,||\cdot||_p)\)成为赋范线性空间.下面证明由范数导出的度量\(d(x,y)\)也是完备的.
			
			先取\(L^p(E)\)中的Cauchy列\(\{f_n\}\),则对任意正数\(\epsilon\),存在正整数\(N\),使得任意\(m,n\ge N\),都有
			\[||f_m-f_n||_p^p=\int_E |f_m-f_n|^p<\epsilon^p.\]
			于是由Chebyshev不等式:
			\[m(|f_m-f_n|^p\ge \epsilon)\le \frac{1}{\epsilon}\int_E |f_m-f_n|^p\le \epsilon^{p-1}.\]
			即\(\{f_n\}\)是按测度Cauchy列.
			
			由于Lebesgue可测函数在测度度量下完备,因此存在\(f\in \mathscr{M}(E),s.t.f_n\to f,n\to \infty.\)再由Fatou:
			\[||f_n-f||_p^p=\int_E |f_n-f|^p\le \liminf\limits_{m\to \infty}\int_E |f_n-f_m|^p<\epsilon^p.\]
			于是得出\(f_n-f\in L^p(E).\)又由\(L^p(E)\)的线性性知:
			\[f=(f-f_n)+f_n \in L^p(E).\]
		\end{proof}
	\end{example}
	\begin{note}
		赋范线性空间的范数性质不仅和空间本身有关,还和等价关系有关.
	\end{note}
	\begin{example}
		证明\(l^p\)空间完备.
		\begin{proof}
			由练习1.1可知\(d_p\)和\(d_\infty\)诱导了相同的拓扑,因此取\(l^p\)空间中的Cauchy列\(\{x_n\}\),任给正数\(\epsilon\),存在正整数\(N\),使得任意\(m,n\ge N\),都有
			\[||x_m-x_n||_p<\epsilon.\]
			于是有
			\[||x_m-x_n||_\infty<\epsilon.\]
			由实数的完备性,\(\{x_n\}\)按坐标收敛于\(x_0\),于是\(x_n\to x_0,n\to \infty\).
			
			如上例,由Fatou知:
			\[||x_n-x_0||_p=(\sum_{i\ge1}|x_n^{(i)}-x_0^{(i)}|^p)^{1/p}\le \liminf_{m \to \infty}(\sum_{i\ge1}|x_n^{(i)}-x_m^{(i)}|^p)^{1/p}\le \epsilon.\]
			于是知\(x_n-x_0\in l^p,\)故
			\[x_0=(x_0-x_n)+x_n\in l^p.\]
		\end{proof}
	\end{example}
	
	\subsection{内积空间和Hilbert正交系}
	我们将线性空间中的内积运算推广到任意向量空间中,具有内积运算的向量空间就是内积空间.正交性是内积空间中重要的性质.
	\begin{definition}[内积空间]
		设\(\mathbb{K}\)为数域,对于\(\mathbb{K}\)上的线性空间\(H\),有\(\Braket{\cdot ,\cdot }:H\times H\longrightarrow \mathbb{K}\),且满足条件:
		\begin{enumerate}
			\item (共轭对称性)\(\Braket{x,y}=\overline{\Braket{y,x}}\)
			\item (第一变元线性性)\(\Braket{\alpha x+\beta y,z}=\alpha \Braket{x,z}+\beta \Braket{y,x}\)
			\item (正定性)\(\Braket{x,x}\ge 0\),且\(\Braket{x,x}=0\)当且仅当\(x=0\).
			
			则称\(\Braket{\cdot ,\cdot }\)是\(H\)中的内积,\(H\)是内积空间.
			
			内积导出的度量是完备的的内积空间称为Hilbert空间.
		\end{enumerate}
	\end{definition}
	\begin{proposition}
		内积可以导出范数:
		\[\sqrt{\Braket{x,x}}=||x||,\]
		范数可以导出度量:
		\[d(x,y)=||x-y||.\]
	\end{proposition}
	\begin{proposition}[平行四边形公式(L1.2.5)]
		设\(H\)是内积空间,\(||\cdot ||\)是内积导出的范数,则对\(H\)中的任意\(x,y\),都有:
		\[||x+y||^2+||x-y||^2=2(||x||^2+||y||^2).\]
	\end{proposition}
	\begin{note}
		只有满足平行四边形公式,空间中的范数才能被内积诱导.(T1.2.3)
	\end{note}
	\begin{property}[内积的连续性]
		若\(x_n\to x,y_n \to y,\)则
		\[\Braket{x_n,y_n}\to \Braket{x,y}.\]
	\end{property}
	\begin{example}
		\(l^p,L^p[a,b]\)空间中只有\(p=2\)时构成Hilbert空间.
		\begin{proof}
			只要证明\(p\ne 2\)时不满足平行四边形公式即可.
			
			\(l^p\):
			
			当\(p \ne 2\),取\(x=(1,1,0,...),y=(1,-1,0,...),\)则\(||x+y||_p=2,||x-y||=2,||x||_p=||y||_p=2^{1/p}\),
			
			若满足平行四边形公式,则
			\[2^{2/p}=2,\]
			也即\(p\ne 2\)时无解.
			
			\(L^p[a,b]\):
			
			不妨设\(a=0,b=1\).取
			\(f=\chi_{[0,\frac{1}{n}]},g=\chi_{[\frac{1}{n},\frac{2}{n}]},\)则有\(||f||_p=||g||_p=(\frac{2}{n})^{1/p}\),
			而\(f+g=2\chi_{[0,\frac{1}{n}]},f-g=2\chi_{[\frac{1}{n},\frac{2}{n}]},||f+g||_p=||f-g||_p=\frac{2}{n^{1/p}}.\)
			
			若平行四边形公式成立,则有
			\[(\frac{2}{n})^{2/p}=2(\frac{1}{n})^{2/p},\]
			\(p\ne 2\)时无解.
		\end{proof}
	\end{example}
	\begin{definition}[正交和正交系]
		\begin{enumerate}
			\item 若\(\Braket{x,y}=0,\)则称\(x,y\)正交.
			\item 设\(\mathfrak{F}\)是Hilbert空间的一族非零向量,若\(\mathfrak{F}\)中任意两个向量都正交,且所有向量范数为1,则称\(\mathfrak{F}\)是\(H\)的一组标准正交系.
		\end{enumerate}
	\end{definition}
	\begin{theorem}[标准正交基的充要条件(T1.3.2)]
		设\(\mathfrak{F}=\{e_\lambda:\lambda \in \Lambda\}\)是Hilbert空间\(H\)中的标准正交系,下面的命题相互等价:
		\begin{enumerate}
			\item 对\(H\)中任意点\(x\),都有\(x=\sum_{\lambda \in \Lambda}\Braket{x,e_\lambda}e_\lambda\),即任意\(H\)中点都能被Fourier展开.
			\item \(\overline{span}\ \mathfrak{F}=H\),即\(\mathfrak{F}\)在\(H\)中稠密.
			\item \(\mathfrak{F}\)完全,即若\(x\in H\)有\(\Braket{x,e_\lambda}=0\ (\forall\lambda \in \Lambda),\)则\(x=0;\)
			\item \(\mathfrak{F}\)完备,即任意\(x\in H,\ ||x||^2=\sum_{\lambda \in \Lambda}|\Braket{x,e_\lambda}|^2\)成立.
		\end{enumerate}
	\end{theorem}
	\begin{note}
		上面(4)中等式为Parseval等式,上面条件给出了是Bessel不等式成立的条件.
		
		Bessel不等式(T1.3.1):
		\[\sum_{\lambda \in \Lambda}|\Braket{x,e_\lambda}|^2 \le ||x||^2.\]
		同时\(H\)中任意\(x\)的Fourier系数最多有可数个不为零.
		Bessel不等式给出了\(x\in H\)的必要条件:\(x\)的Fourier系数平方和级数收敛:
		\[\sum_{\lambda\in \Lambda}|\Braket{x,e_\lambda}|^2<\infty.\]
	\end{note}
	\begin{example}
		证明\(\mathfrak{F}=\{1,...,\sqrt{2}sin\ nx,\sqrt{2}cos\ nx,...\}\)是空间\(L^p[0,2\pi]\)的一组标准正交基,进而得出Parseval等式
		\[||f||^2=\sum_{e\in \mathfrak{F}}|\Braket{f,e}|^2,\]
		从而得出Fourier级数的Riemann-Lebesgue引理:
		\[\lim_{n\to \infty}\int_0^{2\pi}f(x)sin\ nxdx=0,\lim_{n\to \infty}\int_0^{2\pi}f(x)cos\ nxdx=0,\]
		\begin{proof}
			易知\(\mathfrak{F}\)是标准正交系,下面验证\(\overline{span\mathfrak{F}}=L^2[0,2\pi]\)即可.只要证明\(\overline{span\mathfrak{F}}\)在\(C_c^\infty\)中稠密,又由\(C_c^\infty\)在\(L^p[a,b]\)中稠密的事实(见后例),再由稠密性的传递性即可得出\(\overline{span\mathfrak{F}}\)在\(L^p[a,b]\)中稠密.
			
			任取\(f \in C_c^\infty\),对Fourier系数\(a_n(f)\)分部积分可得
			\[a_n(f)=-\frac{a_n(f'')}{n^2}.\]
			于是由Bessel不等式:
			\[|a_n(f'')|^2=|\Braket{f'',\sqrt{2}cos\ nx}|^2\le ||f''||\]
			则有
			\[|a_n(f)|\le \frac{||f''||}{n^2}.\]
			同理可得\(|b_n(f)|\le \frac{||f''||}{n^2}.\)于是部分和
			\[(S_nf)(x)=\frac{1}{2\pi}\int_0^{2\pi}f(t)dt+\sum_{k=1}^n(a_n(f)\sqrt{2}cos\ nx+b_n(f)\sqrt{2}sin\ nx)\]
			满足\((S_nf)(x)\le C+\sup_{x\in[0,2\pi]}\sum_{n\ge 1}\frac{1}{n^2}<\infty.\),即\((S_nf)(x)\)在\([0,2\pi]\)上一致收敛.再由Dini-Lipschitz知道\(S_nf\)点态收敛至\(f\).也就是说\(||S_nf-f||\to 0\),这就证明了\(\overline{span\mathfrak{F}}\)在\(C_c^\infty\)中稠密.
			由于\(\overline{span\mathfrak{F}}=L^2[0,2\pi],\)因此由上述定理的充要条件得出,Parseval等式成立,因此该级数的余项趋于零,即
			\[\lim_{\lambda\to \infty}|\Braket{f,e_\lambda}|^2=\lim_{n\to \infty}|a_n(f)|^2=\lim_{n\to \infty}|b_n(f)|^2=0,\]
			这就证明了Fourier级数的Riemann-Lebesgue引理.
			
			实际上,由Weierstrass第二逼近定理,\(\{z^n=e^{inx}:n\in \mathbb{Z}\}\)张成的三角多项式全体\(\mathcal{P}\)在\(C(\mathbb{T})\)中按照\(L^2\)范数稠密.又由\(C(\mathbb{T})\)按照\(L^2\)范数在\(L^2(\mathbb{T})\)中稠密,由稠密性的传递性,\(\mathcal{P}\)在\(L^2(\mathbb{T})\)中稠密.这提供了上面的另一个证明.
		\end{proof}
	\end{example}
	\begin{definition}[Hilbert空间的同构]
		设\(H,\overline{H}\)都是数域\(\mathbb{K}\)上的Hilbert空间,若存在线性双射\(T:H \longrightarrow \overline{H}\),对任意的\(x,y\in H\)满足内积不变:
		\[\Braket{x,y}=\Braket{Tx,Ty}\]
		则称\(H,\overline{H}\)是保持内积不变同构的Hilbert空间.
	\end{definition}
	\begin{note}
		映射内积不变的性质表明\(T\)还是一个等距.
	\end{note}
	\begin{theorem}[Hilbert空间同构的充要条件[3.6-5]]
		Hilbert空间由标准正交基的势唯一决定,\(H_1\)和\(H_2\)保持内积线性空间同构的充要条件是\(|H_1|=|H_2|.\)
	\end{theorem}
	\begin{note}
		对于每一个Hilbert空间,恰好有一个抽象的实Hilbert空间和一个抽象的复Hilbert空间.换句话说,同一个域上的两个抽象的Hilbert空间只有在势上有所差别.这就把欧几里得空间的情形做了推广.
	\end{note}
	\begin{theorem}[投影定理(T1.3.4)]
		设\(M\)是Hilbert空间\(H\)的一个闭线性子空间,则对于任意的\(x\in H\),存在唯一的\(x_0\in M\),使得\(d(x,M)=||x-x_0||,\)且\(x-x_0 \perp x_0\).
	\end{theorem}
	\begin{note}
		投影定理说明了在用子空间\(M\)的点逼近\(H\)中的点时,最佳逼近点就是\(H\)向\(M\)的投影点,此时投影点距离垂足的距离最小.
	\end{note}
	
	\section{度量空间中的拓扑性质}
	\subsection{基础概念}
	\begin{definition}
		\begin{enumerate}
			\item 开集:由内点构成的集合,集合的内部是包含于集合的最小开集.
			\item 闭集:包含自身所有极限点的集合,即闭集\(X\)满足:
			\[\forall x_0\in X,\exists\{x_n\}\in X,x_n \to x_0.\]
			\item 导集:集合的所有极限点构成的集合.
			\item 闭包:\(\overline{A}=A\bigcup A'\),是包含\(A\)的最小闭集.
			\item 连续映射\(f\):若\(x_n\to x_0,\)则\(f(x_n)\to f(x_0).\)
			若\(f:X\longmapsto Y,\)则\(f\)连续的充要条件是:
			\(Y\)中任意开集(闭集)\(O\)的原像\(f^{-1}(O)\)在\(X\)中是开集(闭集).
		\end{enumerate}
	\end{definition}
	\begin{note}
		\begin{enumerate}
			\item 导集运算符不是幂等的,但是闭包运算符幂等,即\(\overline{\overline{A}}=\overline{A}.\)
			\item \(\overline{A},A'\)都是闭集.
		\end{enumerate}
	\end{note}
	
	\subsection{稠密性和可分性}
	\begin{definition}
		\begin{enumerate}
			\item 稠密性:设\(X\)的子集\(A,B\),若\(A\subset B\)且\(B \subset \overline{A}\),则称\(A\)在\(B\)中稠密.
			\item 可分性:设\(X\)为度量空间,若\(X\)有可数子集在\(X\)中稠密,则称\(X\)是可分的.
		\end{enumerate}
	\end{definition}
	\begin{property}[子空间可分性]
		若\((X,d)\)可分,则任意\(X\)的子集\(A\)有:\((A,d)\)可分.
	\end{property}
	\begin{note}
		稠密性具有传递性:若\(A\)在\(B\)中稠密,\(B\)在\(C\)中稠密,则\(A\)在\(C\)中稠密(按照稠密的范数相同的情况下).
	\end{note}
	\begin{example}
		\(l^\infty,L^\infty[a,b]\)都是不可分的.
	\end{example}
	\begin{example}
		由Weiersreass第一、第二逼近定理,多项式全体\(\mathcal{P}_1\)在\(C[a,b]\)中稠密,三角多项式全体\(\mathcal{P}_2\)在Banach空间\(C(\mathbb{T})\)中稠密.
	\end{example}
	\begin{example}
		\begin{enumerate}
			\item 对直线上的闭区间\([a,b]\)及\(1\le p<\infty\),\(L^p[a,b]\)中的简单函数全体,阶梯函数全体,\(C_c^\infty,C[a,b]\)和多项式全体\(\mathcal{P}\)都是\(L^p[a,b]\)的稠密子集.并且成立关系(\(||\cdot||_p\)范数下):
			\[L^p[a,b]=\overline{SimF}\subset\overline{StaF}\subset\overline{C_c^\infty}\subset\overline{C[a,b]}\subset\overline{\mathcal{P}}.\]
			\item \(L^p(\mathbb{R})(1\le p<\infty)\)中的简单函数全体,阶梯函数全体,\(C_c^\infty(\mathbb{R})\)都是\(L^p(\mathbb{R})\)的稠密子集.
			\item \(L^\infty [a,b]\)只有简单函数全体是稠密子集.
		\end{enumerate}
	\end{example}
	
	\subsection{完备性}
	\begin{definition}
		\begin{enumerate}
			\item (Cauchy列)对度量空间\(X\)中的点列\(\{x_n\},\)任给正数\(\epsilon\),存在正整数\(N\),对任意\(m,n\ge N\),都有\(d(x_m,x_n)<\epsilon,\)就称\(\{x_n\}\)是\(X\)的一个Cauchy列.
			\item 度量空间\(X\)的完备性是指,\(X\)中任意的Cauchy列\(\{x_n\}\)都收敛于\(X\)中的一点\(x_0.\)
		\end{enumerate}
	\end{definition}
	
	\begin{property}
		\begin{enumerate}
			\item 若点列\(\{x_n\}\)有子列\(\{x_{n_k}\}\)收敛于\(x_0\),则点列\(\{x_n\}\)也收敛于\(x_0\).
			\item 度量空间中的收敛点列是Cauchy列,而若度量空间完备则Cauchy列收敛于空间中的点.
			\item 设\(A\)为度量空间\((X,d)\)的一个子集,子空间\((A,d)\)完备的充要条件是\(A\)是\(X\)的一个闭子集.(L1.2.1)
			\item 不完备空间可以具有完备子空间(如\(\mathbb{Q}\)的子空间\(\mathbb{Z}\)完备).
		\end{enumerate}
	\end{property}
	\begin{corollary}[T1.4.2]
		有限维赋范线性空间完备.
	\end{corollary}
	
	\begin{theorem}[闭球套定理和度量空间的完备性等价(T1.4.4,1.4.5)]
		闭球套定理成立的度量空间完备,完备度量空间成立闭球套定理.
		
		\textit{闭球套定理}
		
		设\(X\)是完备的度量空间,设\(B_n=\{x\in X:d(x,x_n) \le \epsilon_n\}\)是一列单调下降的闭球:
		\[B_{n+1} \subset B_n,\]
		若球的半径\(\epsilon_n \to 0,\)则存在唯一的\(x\in X,\)使得\(x\in B_n,\forall n\in \mathbb{N}.\)
	\end{theorem}
	\begin{corollary}
		\begin{enumerate}
			\item 等价范数导出相同的拓扑.
			\item 等价范数下的同一个度量空间中的Cauchy列等价.
			\item 对于一个有限维的赋范线性空间,该空间上的所有范数都是等价的.
		\end{enumerate}
	\end{corollary}
	\begin{proof}
		\begin{enumerate}
			\item 设范数\(||\cdot||_X,||\cdot||_Y\)等价,即对任意的\(x\in X,\)、存在正数\(c,C\)使得
			\[c||x||_X\le ||x||_Y\le C||x||_X.\]
			则当\(||x_n-x_0||_X\to 0,\)就有\(||x_n-x_0||_Y\to 0.\)反之也有当\(||x_n-x_0||_Y\to 0,\)就有\(||x_n-x_0||_X\to 0.\)这就证明了等价范数诱导了相同的拓扑.
			\item 由上面的证明可知两个等价范数下的Cauchy列等价.
			\item 有限维的赋范线性空间完备,因此任取Cauchy列\(\{x_n\}\)都收敛于\(x\in X,\)也即Cauchy列等价,因此所有范数等价.
		\end{enumerate}
	\end{proof}
	
	\begin{example}
		证明\(C[a,b]\)完备.
		\begin{proof}
			设\(\{f_n\}\)是\(C[a,b]\)中的Cauchy列,于是对任意正数\(\epsilon\),存在正整数\(N\),使得任意\(m,n \ge N\)时有
			\[\max_{a \le x \le b}|f_n(x)-f_m(x)|<\epsilon.\]
			也就是说\(\{f_n(x)\}\)是\(\mathbb{R}\)中的Cauchy列.由实数的完备性,知道\(\{f_n(x)\}\)点态收敛于函数\(f(x).\)
			
			下面证明\(f(x)\in C[a,b].\)只要证明\(\{f_n(x)\}\)一致收敛于\(f(x)\)即可.由前式,令\(m \to \infty\),由于点态收敛性,只要\(n \ge N\),就有
			\[|f_n(x)-f(x)|<\epsilon,\forall x \in [a,b].\]
			此即\(\{f_n(x)\}\)一致收敛于\(f(x)\),故有\(f(x)\in C[a,b].\)
		\end{proof}
		\begin{note}
			证明度量空间完备性分为两步:第一步写出Cauchy列,找出Cauchy列的极限点(本例通过实数的完备性将\(C[a,b]\)的Cauchy列转化为实数Cauchy列处理,并得出其极限);第二部证明该极限点仍在度量空间内(本例由一致连续性得出结论).
		\end{note}
	\end{example}
	
	\begin{example}
		证明\((c_0,||\cdot||_\infty)\)完备.
		\begin{proof}
			\(c_0\)是\(l^\infty\)的线性子空间,并且\(||\cdot||_\infty\)也给出了\(c_0\)的一个范数,因此由本节性质3,只要证明\(c_0\)的闭性,即证明\(c_0\)中的极限点还在\(c_0\)中即可.
			
			设\(c_0\)中有Cauchy列\(\{x_n\},x_n=(x_1^{(n)},\dots,x_k^{(n)},\dots),\)易知存在\(x_0 \in l^\infty,\)使得\(||x_n-x_0||_\infty\to 0.\)只要证明\(x_0\in c_0\)即可.
			
			对于Cauchy列\(\{x_n\}\in c_0,x_n\to 0,n \to \infty.\)对任意正数\(\epsilon\),存在正整数\(j\),使得任意\(m \ge j\),就有\(|x_m|< \epsilon.\)同时有\(||x_m-x_0||<\epsilon\)于是对于\(x_0\)的任意坐标分量\(x_j^{(0)}\)有:
			\[|x_j^{(0)}|\le |x_j^{(m)}-x_j^{(0)}|+|x_j^{(m)}|\le ||x_m-x_0||_\infty+\epsilon< 2\epsilon.\]
			由于按坐标收敛在\(d_\infty\)下等价于按范数\(||\cdot||_\infty\)收敛,于是知道\(x_0\in c_0.\)
		\end{proof}
	\end{example}
	
	\subsection{列紧性和紧度量空间}
	\begin{definition}[(相对)列紧集]
		\begin{enumerate}
			\item 相对列紧集:设有度量空间\(X\),\(A\)是\(X\)中的相对列紧集,当任意\(A\)中的点列都有收敛于\(X\)中的子列.
			\item 列紧集:闭的相对列紧集,即相对列紧集的任意点列都有子列收敛于\(A\).
		\end{enumerate}
	\end{definition}
	\begin{property}[(T1.6.2,T1.6.3,E1.6.3)]
		\begin{enumerate}
			\item \(A\)完全有界且\(X\)完备\(\Rightarrow{}A\)相对列紧(Hausdorff)
			\item \(A\)相对列紧且\(X\)闭\(\Rightarrow{}A\)列紧\(\Leftrightarrow A\)紧\(\Leftrightarrow A\)的开覆盖族存在有限开覆盖
			\item \(A\)相对列紧\(\Rightarrow{}A\)完全有界\(\Rightarrow{}A\)有界
		\end{enumerate}
	\end{property}
	\begin{note}
		度量空间中的列紧集等价于紧集.
	\end{note}
	\begin{theorem}[有限维和无穷维空间的本质区别(T1.6.1)]
		\begin{enumerate}
			\item 无穷维空间\(X\)的单位闭球非相对列紧.
			\item 有限维的赋范线性空间的有界集完全有界.
		\end{enumerate}
	\end{theorem}
	\begin{note}
		有限维赋范线性空间完备,因此通过其有界集完全有界可以得出有界集相对列紧,又由空间的完备性得知空间的闭性,于是有界集列紧.在度量空间中列紧性和紧性相同,于是知道Banach空间中的有界集等价于紧集.因此得知有限维赋范线性空间的单位闭球相对列紧.
	\end{note}
	\begin{theorem}[紧集上的连续映射性质(T1.6.8,C1.6.3,C1.6.4,,T1.6.9)]
		\begin{enumerate}
			\item 设\(K\)为紧集,\(f\)为\(K\)上的连续映射,则\(f(K)\)紧.
			\item (紧集上的连续映射有界)紧集上的连续映射像集有界,并且上下确界可达.
			\item 紧集上的连续双射\(f\)若满足逆\(f^{-1}\)连续,则\(f\)为拓扑同胚.
			\item 设\(X\)为紧度量空间,若\(f\)为\(X\)上的连续函数,则\(f\)一致连续.
			\item (Arzela-Ascoli)\ \(C(X)\)中的子集\(A\)是相对列紧集的充要条件是\(A\)为等度连续的有界集.
		\end{enumerate}
	\end{theorem}
	
	
	\chapter{线性泛函}
	\begin{introduction}
		\item 线性算子
		\item 有界性、连续性
		\item 连续泛函
		\item Hahn-Banach延拓定理
		\item 对偶空间和几种空间下连续泛函的表现形式
		\item 二次对偶空间和自反性
		\item 弱拓扑和弱*拓扑
	\end{introduction}
	\section{赋范线性空间上的线性算子}
	\subsection{线性算子}
	保持线性性(群同态)的映射就是线性算子.
	\begin{definition}[线性算子]
		设\(\mathbb{K}\)是数域,\(X,Y\)是\(\mathbb{K}\)上的线性空间,\(\mathcal{D}\)是\(X\)的子集.\(\mathcal{R}\)是\(Y\)的子集.若有映射\(T:\mathcal{D}\rightarrow{\mathcal{R}}\),满足线性性:
		\[T(\alpha x+\beta y)=\alpha T(x)+\beta T(y),\ \forall \alpha,\beta \in \mathcal{D},\]
		就称\(T\)是线性算子,其中\(\mathcal{D,R}\)分别是\(T\)的定义域和值域.值域\(\mathcal{R}=T(\mathcal{D)}\).又称\(ker\ T=\{x\in \mathcal{D}:T(x)=0\}\)为\(T\)的核(零空间).
		
		特别地,若\(\mathcal{R}=\mathbb{K}\)时,称\(T\)为\(X\)上的线性泛函,当\(\mathcal{R}=X\)时,称\(T\)为\(X\)上的线性算子.
	\end{definition}
	容易看出\(T0=0\).线性算子本质就是映射,因此也继承了映射的几条性质:
	\begin{property}[2.6-9]
		设\(T:X\rightarrow{Y}\),定义域\(X\)是线性空间.则:
		\begin{enumerate}
			\item 值域\(T(X)\)是线性空间;
			\item 零空间\(ker\ T\)是线性空间;
			\item 若\(dim\ X=n<\infty,\)则\(dim\ TX\le n.\)
		\end{enumerate}
		逆算子的性质(2.6-10):
		\begin{enumerate}
			\item 逆算子\(T^{-1}\)存在的充要条件是:\(Tx=0\)蕴含\(x=0\);
			\item 若逆算子\(T^{-1}\)存在,则\(T^{-1}\)也是线性算子;
			\item 若逆算子\(T^{-1}\)存在,则由于\(dim\ X\le dim\ TX\),且由对称性,有\(dim\ TX\le dim\ X\),故有\(dim\ X= dim\ TX\).
		\end{enumerate}
	\end{property}
	\begin{example}
		可以验证Fourier系数映射\(\tau:L^1(\mathbb{T})\rightarrow{c_0(\mathbb{Z})},\ \tau(f)=\{\hat{f}(n)\}\),是线性算子,其中\(\hat{f}(n)=\Braket{f,e^{in\theta}}\).再由Riemann-Lebesgue引理,\(\lim_{n\to \infty}{\Braket{f,e^{in\theta}}}=0\)可知,\(\lim_{n\to \infty}\hat{f}(n)=0,\)即\(\tau(f)\in c_0(\mathbb{Z})\).
	\end{example}
	\begin{example}
		Fourier变换\(\mathcal{F}:L^1(\mathbb{R})\rightarrow{c_0(\mathbb{R})}\)形为
		\[\mathcal{F}(f)(t)=\int_\mathbb{R}f(x)e^{-i2\pi tx}dx,\ f\in L^1(\mathbb{R}).\]
		当\(t_n\to t,\)有\(f(x)e^{-i2\pi t_nx}\to e^{-i2\pi tx}\)(点态收敛).而\(|f(x)e^{-i2\pi tx}|=|f(x)|\in L^1(\mathbb{R})\),故由Lebesgue控制收敛:
		\[\int_\mathbb{R}f(x)e^{-i2\pi t_nx}dx\to \int_\mathbb{R}f(x)e^{-i2\pi tx}dx,\]
		即\(\mathcal{F}(f)(t_n)\to \mathcal{F}(f)(t),\)故\(\mathcal{F}(f)(t)\)连续.而由Riemann-Lebesgue,当\(|t|\to \infty,\ \mathcal{F}(f)(t)\to 0.\)故\(\mathcal{F}(f)\in c_0(\mathbb{R}).\)
	\end{example}
	
	\subsection{线性算子的有界性、连续性}
	\begin{theorem}[连续性和有界性(2.1.1)]
		设\(T:X\rightarrow{Y}\)是线性算子,则下列命题等价:
		\begin{enumerate}
			\item \(T\)是连续映射,即\(x_n\to x\)蕴含\(Tx_n\to T_x\)或其拓扑表述.
			\item \(T\)在\(X\)任意一点连续.
			\item (有界性)存在正数\(M\),使得\(||Tx||_Y\le M||x||_X.\)
		\end{enumerate}
	\end{theorem}
	\begin{note}
		有界性一条说明,对于有界的\(X\),其值域\(T(X)\)仍是有界的.
	\end{note}
	并不是所有算子都是有界的,比如求导算子就是无界的.
	\begin{example}
		设\(X=C^{(1)}[a,b],\)范数定义为\(C[a,b]\)的范数。即\(||x||=\max_{a\le t\le b}x(t)\).定义求导算子\(D:X\to C[a,b]\):
		\[D(x)(t)=\frac{d}{dt}x(t),\ \forall x \in X.\]
		则\(D\)无界.
		\begin{proof}
			只要取\(x_n(t)=e^{-n(t-a)},\ x_n(t)\in X\),可以知道\(||x||=1,\)但是\(D(x_n)(t)=-ne^{-n(t-a)}\),有\(||D(x_n)(t)||=n\to \infty\),这就说明了\(D\)是无界算子.
		\end{proof}
	\end{example}
	Think outside of the box,实际上\(X\)上的有界算子全体也能形成一个线性空间,记为\(X'\),称其为\(X\)的对偶空间.实际上\(X'\)也成为线性空间,而且可以在上面定义范数,使其成为赋范线性空间.特别地,当\(T\)是有界线性泛函时,记\(X'\)为\(X^*\),称其为\(X\)的(代数)对偶空间.我们后面提到的对偶空间都是指有界线性泛函构成的空间.
	\begin{theorem}[对偶空间(2.1.2)]
		设\(X,Y\)是赋范线性空间,\(X'\)是\(X\)到\(Y\)的有界线性算子全体.若
		\begin{enumerate}
			\item 对\(T\in X'\)定义范数\[||T||=\sup_{x\ne 0}\frac{||Tx||}{||x||}\]
			\item 设\(A,B\in X',\ \alpha\in \mathbb{K}\),定义算子运算:
			\begin{enumerate}
				\item \((A+B)(x)=Ax+Bx\);
				\item \((\alpha A)x=\alpha Ax\).
			\end{enumerate}
			则\(X'\)成为赋范线性空间.
			
			同时,若\(Y\)是Banach空间,则\(X'\)也是Banach空间.
		\end{enumerate}
	\end{theorem}
	\begin{note}
		\(T\)的范数\(||T||\)还能表示为
		\[||T||=\sup_{||x||=1}||Tx||=\sup_{||x||<1}||Tx||.\]
		
		由上确界性质易知\(||Tx||\le ||T||\cdot||x||,\ \forall x\in X.\)我们可以把\(||T||\)看作在单位球\(||x||=1\)上,像\(||Tx||\)的最大扩展系数.
		
		同时,\(T\)有界的充要条件是\(||T||<\infty.\)
	\end{note}
	\begin{proof}
		先证明\(||\cdot||\)满足范数定义.
		
		正定性显然.而\(||\alpha T||=\sup_{||x||=1}||\alpha Tx||_Y=|\alpha|\sup_{||x||=1}||Tx||.\)
		
		三角不等式:\(||T_1+T_2||=\sup_{||x||=1}||(T_1+T_2)x||_Y=\sup_{||x||=1}||T_1x+T_2x||_Y\le \sup_{||x||=1}||T_1x||_Y+\sup_{||x||=1}||T_2x||_Y=||T_1||+||T_2||.\)
		
		再证明\(X'\)是线性空间设\(A,B\in X'\).
		
		\(||(A+B)x||\le ||Ax||+||Bx||<\infty;\ ||\alpha Ax||=|\alpha|\cdot||Ax||<\infty.\)
		因此\(A+B\in X',\ \alpha A\in X'.\)线性性得证.
		
		下面证明当\(Y\)是Banach空间时,\(X'\)也是Banach空间.
		
		取\(X'\)中的Cauchy列\(\{T_n\}\),则对于任意正数\(\epsilon\),存在正整数\(N\),使得任意的\(m,n\ge N,\)都有\(||T_m-T_n||<\epsilon.\)因此\(||T_mx-T_nx||=||(T_m-T_n)x||\le ||T_m-T_n||\cdot||x||\le\epsilon||x||.\)于是知道\(T_nx\)是\(Y\)中的Cauchy列.由于\(Y\)完备,因此\(\{T_nx\}\)存在极限,记为\(Tx\).下面证明\(T\in X'.\)
		
		由于\(||(T-T_n)x||=||Tx-T_nx||=\lim_{m\to \infty}||T_mx-T_nx||\le \epsilon||x||,\)因此\(T-T_n\in X'\),故\(T=T_n-(T_n-T)\in X'\),即\(X'\)完备.
	\end{proof}
	\begin{note}
		\begin{enumerate}
			\item \(X^*\)的完备性只和\(Y\)有关,而和\(X\)无关.
			\item 由实数完备性,元素为线性泛函的对偶空间\(X^*\)是Banach空间.
		\end{enumerate}
	\end{note}
	除了线性运算,我们还能再为算子定义乘法运算,即设有赋范线性空间\(X,Y,Z,\ A\in X^*,B\in Y^*\),规定
	\[(BA)x=B(Ax),\]
	则\(BA:X\to Z,\)且\(||BA||\le ||B||\cdot||A||.\)
	
	如果有\(T:X\to X,\),则由上面定义的乘法,我们有\(||T^n||\le||T||^n.\)
	
	\section{有界线性泛函}
	\subsection{线性泛函和核}
	线性泛函是最简单的线性算子,具有很多特殊的性质.线性泛函的很多性质都由它的核决定.
	
	设\(A\)是\(B\)的子空间,则\(A\)的余维数定义为:\(codim\ A=dim\ B/A\).其中\(B/A\)是商空间.
	\begin{theorem}[线性泛函和核的余维数(2.2.0)]
		设\(f\)是线性空间\(V\)上的线性泛函,则 \(codim\ ker\ f=1.\)
		
		反之,\(V'\)是\(V\)的一个余维数为1的线性子空间,则存在\(V\)上的线性泛函\(f\)使得\(ker\ f=V'\).
	\end{theorem}
	\begin{note}
		定理揭示了余维数为1的子空间和线性泛函的核的一一对应关系.
	\end{note}
	\begin{theorem}[具有闭核的线性泛函有界]
		设线性算子\(T:X\to Y.\)若\(T\)有界,则\(ker\ T\)闭,但是反之则不然.
		\begin{proof}
			线性算子有界等价于连续,故\(ker\ T=f^{-1}\{0\}\)仍为闭集,因为\(\{0\}\)闭.反之考虑求导算子,\(ker\ D\}\)为常数集,显然是闭的,但是\(D\)无界.
		\end{proof}
		然而,对于线性泛函\(f,\ f\)有界的充要条件是\(ker\ f\)闭.
	\end{theorem}
	\begin{proof}
		必要性在上面已经证过,下面证充分性.
		
		反证法.设\(ker\ f\)闭,但是\(f\)无界,即\(||f||=\infty.\)于是可以取点列\(\{x_n\},\ ||x_n||=1,\ |f(x_n)|>n.\)此时设
		\[y_n=\frac{x_n}{f(x_n)}-\frac{x_1}{f(x_1)},\]
		可知\(y_n\in ker\ f\).然而由于\(|\frac{x_n}{f(x_n)}|\to 0\),于是有\(y_n\to -\frac{x_1}{f(x_1)}\),即
		\[\lim_{n\to \infty}y_n=-\frac{x_1}{f(x_1)},\]
		但是由于\(f(-\frac{x_1}{f(x_1)})=-1\ne 0,\)故\(y_n\to y\notin ker\ f\),这与\(ker\ f\)的闭性矛盾.故\(f\)有界.
	\end{proof}
	
	\subsection{特定空间上的连续线性泛函的表示形式和空间的同构}
	实际上,在特定的空间(比如Hilbert空间,\(L^p[a,b]\))上定义的连续线性泛函具有固定的形式,而且每个形式都与另一个空间中的某个元素形成一一对应.于是我们着重研究什么样的连续线性泛函能和另一个空间中的元素形成一一对应关系.
	\begin{definition}[同构]
		设有赋范线性空间\(X,Y\)和映射\(T:X\to Y.\)
		\begin{enumerate}
			\item 若\(||Tx||_Y=||x||_X,\ \forall x\in X\),则称\(T\)是保范算子.
			\item 若\(T\)既是保范的,又是线性映射,而且还是\(X,Y\)上的一一对应,则称\(T\)是\(X,Y\)上的(保范)同构映射.
			\item 若\(X,Y\)上存在映射\(T:X\to Y\)是保范同构映射,就称\(X,Y\)是同构的记为\(X\cong Y\).
		\end{enumerate}
	\end{definition}
	下面来看几种特定赋范线性空间上的连续线性泛函的表现形式.
	\begin{theorem}[F.Riesz连续线性泛函表示(2.2.2-)]
		\begin{enumerate}
			\item Hilbert空间:设\(H\)是一个Hilbert空间,\(F\)是\(H\)上的连续线性泛函,则存在唯一的\(y\in H,\)使得
			\[F_y(x)=\Braket{x,y},\ \forall x\in H,\]
			且有\(||y||=||F_y||.\)
			\item \(l^p\)空间:设\(1\le p<\infty,\ \frac{1}{p}+\frac{1}{q}=1.\)若\(f\)是\(l^p\)上的线性连续泛函,则存在唯一的\(\eta\in l^q\),使得
			\[f_\eta(x)=\sum_{i=1}^\infty x_i\eta_i,\ \forall x\in l^p.\]
			且有\(||\eta||_q=||f_\eta||.\)这说明了\((l^p)^*\cong l^q.\)
			\item \(L^p[a,b]\):设\(1\le p<\infty,\ \frac{1}{p}+\frac{1}{q}=1.\)若\(f\)是\(L^p[a,b]\)上的线性连续泛函,则存在唯一的\(\beta\in L^q[a,b]\),使得
			\[f_\beta(x)=\int_a^bx(t)\beta(t)dt,\ \forall x(t)\in L^p[a,b],\]
			且有\(||\beta||_q=||f_\beta||.\)这说明\((L^p[a,b])^*\cong L^q[a,b].\)
			\item \(C[a,b]\):若\(f\)是\(C[a,b]\)上的线性连续泛函,则存在唯一的\(g\in V_0[a,b]\),使得
			\[f_g(x)=\int_a^bx(t)dg(t),\]
			且\(||g||_{V_0}=||f_g||.\)这说明\(C[a,b]\cong V_0[a,b]\).
		\end{enumerate}
	\end{theorem}
	\begin{note}
		我们发现,设赋范线性空间\(X,Y\)和连续线性泛函\(f:X\to Y,\)这些泛函具有固定的表现形式,而且还和另一个空间中的点形成了一一对应,并且泛函的范数和这些点的范数相等,这就给出了对偶空间到另一个空间的同构.
	\end{note}
	\subsection{有限维的例子}
	\begin{theorem}[有限维线性算子和矩阵]
		有限维线性空间上选定基后,线性算子和矩阵一一对应.由矩阵的运算性质,有限维赋范线性空间中的线性算子连续.
	\end{theorem}
	\begin{theorem}[对偶空间维数(2.9-1)]
		设\(X\)是\(n\)维向量空间,其基为\(\{e_i\}_{i=1}^n\),则\(\{f_i\}_{i=1}^n\)是对偶空间\(X^*\)的一组基.因此\(dim\ X=dim\ X^*=n.\)
		
		其中\(f_i(e_j)=\delta_{ij},\ \delta_{ij}=\begin{cases}
			0,i\ne j \\
			1,i=j.
		\end{cases}\)
	\end{theorem}
	\begin{proof}
		先证明线性无关性:设\(f(x)=\sum_{i=1}^nk_if_i(x)=0,\)代入\(x=e_j\),则有\(f(e_j)=\sum_{i=1}^nk_if_i(e_j)=k_j=0.\)
		
		再证明任意\(f\in X^*\)都能被\(\{f_i\}_{i=1}^n\)线性表示.设\(f(x)=\sum_{i=1}^nk_if(e_i),\)则\(f_j(x)=\sum_{i=1}^nk_if_j(e_i)=k_j\),即\(f(x)=\sum_{i=1}^nf_i(x)f(e_i)\).
	\end{proof}
	\begin{note}
		注意这里\(f,f_i\in X^*\),而\(f(e_j),f(x)\in \mathbb{K}\).
	\end{note}
	我们还有下列引理:
	\begin{lemma}[2.9-2]
		设\(X\)是有限维向量空间.若\(x_0\in X\)满足对任意的\(f\in X^*\)都有\(f(x_0)=0\),则\(x_0=0.\)
	\end{lemma}
	\begin{proof}
		设\(x_0=\sum_{i=1}^nk_ie_i\),则\(f(x_0)=\sum_{i=1}^nk_if(e_i)=\sum_{i=1}^nk_i\sum_{j=1}^nl_jf_j(e_j)=0\)对于任何\(f\in X^*\)成立,因此\(k_i=0,\ x_0=0.\)
	\end{proof}
	
	
	\section{Hahn-Banach延拓定理}
	一般来说,在延拓问题中考虑的数学对象(比如映射)本来定义在给定集合\(X\)的一个子集\(Z\)上,我们希望把它从\(Z\)延拓到整个\(X\)上,并且要求原对象的某些性质在延拓后能够继续保留.在Hahn-Banahn延拓定理中,被延拓的对象是定义在向量空间\(X\)的子空间\(Z\)上的线性泛函\(f\),还要求这个泛函具有一定的有界性质,而这个有界性质是用次线性泛函来描述的.线性空间\(L\)上的次线性函数\(p\)满足次可加性和正齐次性:
	\[p(x+y)\le p(x)+p(y),\ \forall x,y\in L;\]
	\[p(tx)=tp(x),\ \forall x\in L,t\ge0.\]
	这个定理保证了赋范空间充分地配备了有界线性泛函,并且使我们有可能得到足够的对偶
	空间理论.这里只介绍原定理的推论.
	\begin{theorem}[Hahn-Banach(2.3.1)]
		设实赋范线性空间\(L\)有线性子空间\(L_0\),若\(f_0\)是\(L_0\)上的有界线性泛函,则存在\(L\)上的线性泛函\(f\),使得\(f\)是\(f_0\)的延拓,满足\(f|_{L_0}=f_0,\ ||f||=||f_0||.\)
	\end{theorem}
	下面来看看Hahn-Banach延拓定理的应用.
	\begin{corollary}
		对赋范线性空间\(X\)中的非零元素\(x_0\),存在\(f\in X^*\),使得\(||f||=1,f(x_0)=||x_0||.\)
	\end{corollary}
	\begin{proof}
		作\(x_0\)张开的一维线性空间\(\mathbb{K}x_0\)上的泛函\(f_0\):
		\[f_0(ax_0)=|a|\cdot||x_0||,\]
		则\(||f_0||=\sup_{||x||=1}|f_0(x)|=1\).故由Hahn-Banach延拓定理:存在\(f\in X^*\),满足
		
		\(f|_{\mathbb{K}x_0}(x_0)=f_0(x_0)=||x_0||,\ ||f||=||f_0||.\)
	\end{proof}
	\begin{corollary}[分离性1]
		对赋范线性空间\(X\)中的不同两个元素\(x_1,x_2\),存在\(f\in X^*\),使得\(f(x_1)\ne f(x_2)\),即\(X^*\)分离\(X\)中的点.
	\end{corollary}
	\begin{proof}
		由于\(x_1-x_2\ne0,\)由上一条推论知道\(f(x_1)-f(x_2)=f(x_1-x_2)=||x_1-x_2||\ne0.\)
	\end{proof}
	\begin{corollary}[对偶]
		对赋范线性空间\(X\)中的元素\(x\),存在\(f\in X^*\)满足
		\[||x||=\max_{||f||=1}|f(x)|.\]
	\end{corollary}
	\begin{proof}
		易知\(|f(x)|\le||f||\cdot||x||=||x||.\)而由上述推论知这样的\(f\)是存在的,故等号成立.
	\end{proof}
	\begin{corollary}[分离性2]
		设\(M\)是赋范线性空间\(X\)的闭线性子空间,若有\(x \notin M,\)则存在\(f\in X^*,\)使得\(||f||=1,\ f(M)=0,\ f(x)=d(x,M).\)这说明了\(X^*\)可以分离\(X\)中的闭线性子空间和点.
	\end{corollary}
	
	\section{二次对偶空间和自反}
	\(X\)的对偶空间\(X^*\)的对偶空间\(X^{**}\)称为\(X\)的二次对偶空间,它与\(X\)具有很大的关系,特别是当\(X\)自反时.
	固定\(z\),我们可以定义映射\(J:X\to X^{**},\ x\to g(f)=g_x(f)=f(x)\)称为典范映射(自然嵌入).可以证明\(J\)是线性泛函.事实上,\(J(\alpha x+\beta y)=g_{\alpha x+\beta y}(f)=f(\alpha x+\beta y)=\alpha f(x)+\beta f(y)=\alpha J(x)+\beta J(y).\)
	
	同时,\(J\)还是保范的,即\(||J(x)(f)||=||g_x(f)||=||x||\).因为\(||g_x||=\sup_{f\ne 0,f\in X^*}\frac{|g_x(f)|}{||f||}=\sup_{||f||=1}|f(x)|=||x||\)(推论2.3).
	\begin{note}
		典范映射\(J:X\to X^{**},x\to g_x(f)=f(x)\)中,注意\(x\in X\)是固定的量,而变量是\(f\in X^*.\)典范映射\(J(x):X^*\to \mathbb{K},\ J(x)(f)=f(x)\).(其实写成\(x(f)\)可能会更好些)
	\end{note}
	\begin{definition}[自反]
		若典范映射\(J:X\to X^{**}\)是同构,称\(X\)自反.此时以等价类\("\cong"\)代替\("="\).
	\end{definition}
	\begin{note}
		事实上存在同构\(J:X\to X^{**}\)不是满射.
	\end{note}
	\begin{corollary}
		若\(X\)自反,则\(X^*\)自反.事实上\((X^*)^{**}=(X^{**})^*=X^*.\)
	\end{corollary}
	\begin{example}
		设\(1\le p <\infty,\ \frac{1}{p}+\frac{1}{q}=1.\)则有\((L^p[a,b])^{**}=L^p[a,b],\)存在\(J=I\)为同构,因此\(L^p[a,b]\)自反.同理有\(l^p\)自反.
	\end{example}
	不过也有非自反的例子,比如\(L^1[a,b]\)就不是自反的.为了证明这一点,我们要证明下面的定理:
	\begin{theorem}[2.5.5]
		设\(X\)是赋范线性空间,若\(X^*\)可分,则\(X\)可分.
	\end{theorem}
	有了这个定理,我们来证明\(L^1[a,b]\)不是自反的.反证法:若\(L^1[a,b]\)自反,即\((L^1[a,b])^{**}\)=\(L^1[a,b]\),故\((L^1[a,b])^{**}\)可分,于是\((L^1[a,b])^*\)=\(L^\infty[a,b]\)可分,矛盾.
	\begin{corollary}[对偶]
		若\(X\)自反,对任意的\(f\in X^*,\)都存在\(x\in X,\)使得\(||x||=1,\ f(x)=J(x)(f)=||f||.\)
	\end{corollary}
	\begin{note}
		若\(X\)自反,设\(f_1\ne f_2\),则存在\(x\in X,\ ||x||=1,\)使得\(f_1(x)-f_2(x)=(f_1-f_2)(x)=J(x)(f_1-f_2)=||f_1-f_2||\ne 0,\)即\(X^{**}\)分离\(X^*\)的点.
	\end{note}
	\begin{corollary}
		\begin{enumerate}
			\item 自反的线性空间完备.事实上,设赋范线性空间\(X\)自反,则\(X=X^{**}=(X^*)^*\)完备.
			\item Hilbert空间自反.
		\end{enumerate}
	\end{corollary}
	
	\section{弱拓扑和弱\(^*\)拓扑}
	我们知道有限维空间和无限维空间的本质区别是,在赋范线性空间\(X\)的范数诱导的拓扑下,有限维空间中的闭单位球是紧的,而无限维的闭单位球不是.为了解决这个问题,我们引入比范数诱导的拓扑更弱的条件,也就是弱拓扑的概念.
	\begin{definition}[弱收敛和弱\(^*\)收敛]
		设有赋范线性空间\(X\)和其对偶空间\(X^*\).
		\begin{enumerate}
			\item 若点列\(\{x_n\}\in X\)对于任何\(f\in X^*\)都有\(f(x_n)\to f(x)\),则称\(\{x_n\}\)弱收敛于\(x,\ x\in X.\)记作\(x_n\xrightarrow{w}x.\)
			\item 若点列\(\{f_n\}\in X^*\)对于任何\(x\in X\)都有\(f_n(x)\to f(x)\),则称\(\{f_n\}\)弱\(^*\)收敛于\(f\in X^*\).记作\(f_n\xrightarrow{w^*}f\).
		\end{enumerate}
	\end{definition}
	\begin{note}
		由于\(X^*\)分离\(X\)中的点,因此弱收敛极限唯一;又由于\(X^{**}\)分离\(X^*\)中的点,因此弱\(^*\)收敛极限唯一.
	\end{note}
	\begin{proposition}[依范数收敛(强收敛)和弱收敛的关系]
		\begin{enumerate}
			\item 强收敛蕴含弱收敛,反之不然.事实上若\(\{x_n\}\)强收敛于\(x\in X\),则对任意\(f\in X^*,\ |f(x_n)-f(x)|\le ||f||\cdot ||x_n-x||\to 0\),反之考虑\(X=L^2[0,2\pi],\)由Riemann-Lebesgue:\(f(e^{in\theta})-f(0)=\int_{0}^{2\pi}f(x)e^{in\theta}dx\to 0,\ n\to \infty.\)即\(e^{in\theta}\xrightarrow{w}0.\)然而\(||e^{in\theta}||_2-||0||_2=\sqrt{2\pi}\ne 0,\)即\(e^{in\theta}\)不强收敛于0.
			\item 强\(^*\)收敛蕴含弱\(^*\)收敛,反之不然.若\(f_n\)依\(X^*\)范数收敛于\(f\in X^*\),则\(|f_n(x)-f(x)|=|(f_n-f)x|\le ||f_n-f||\cdot ||x||\to 0,\ n\to \infty.\)反之考虑\(\{e_n\}\in l^\infty,\ x\in l^1,\)则\(|e_n\cdot x-0|=|x_n|\to 0\),即\(e_n\xrightarrow{w^*}0\).但\(||e_n||_{l^\infty}-0=1\ne 0,\)因此\(\{e_n\}\)不强收敛于0.
			\item 赋范线性空间\(X\)中点列\(x_n\xrightarrow{w}x\)等价于二次对偶空间中\(X^{**}\)点列\(Jx_n\xrightarrow{w^*}Jx\).实际上二次对偶空间\(X^{**}\)的点列\(Jx_n\xrightarrow{w^*}Jx\)等价于任意\(f\in X^*,Jx_n(f)\to Jx(f),\)而这就是\(f(x_n)\to f(x)\).例如,\(x^{(n)}\xrightarrow[c_0]{w}x\)和\(x^{(n)}\xrightarrow[l^\infty]{w^*}x\)都是指对于任意的\(\eta \in l^1,\ x^{(n)}^T\eta\to x^T\eta\).
			\item 弱收敛蕴含弱*收敛.
			\item 若\(X\)自反,则\(X^*\)的弱收敛和弱*收敛等价.
		\end{enumerate}
	\end{proposition}
	\begin{note}
		有限维赋范线性空间中,弱收敛等价于强收敛.
	\end{note}
	这里省略弱拓扑、弱*拓扑、弱紧、弱*紧的定义,见定义2.5.3.
	
	下面介绍弱拓扑中的重要定理.
	\begin{theorem}[Mazur(2.5.7)]
		设\(M\)是赋范线性空间\(X\)中的凸子集,则\(M\)的范数拓扑闭包等于其弱拓扑闭包.
	\end{theorem}
	\begin{theorem}[Banach-Alaoglu(2.5.8)]
		赋范线性空间\(X\)的对偶空间\(X^*\)中的闭单位球是弱*紧的.
	\end{theorem}
	
	
	\chapter{有界算子的基本定理}
		\begin{introduction}
		\item Baire纲定理
		\item 几个有界算子的基本定理
	\end{introduction}
	\section{Baire纲定理}
	相对于稠密集的概念,我们引入疏朗集的概念来衡量一个集合中子集元素数量的多少.
	\begin{definition}[疏朗集]
		设\(A\)是度量空间\(X\)的一个子集,若\(\overline{A}\)不含有\(X\)的非空开集,即\(\overline{A}\)没有内点,则称\(A\)是\(X\)中的疏朗集.
		
		\(A\)是疏朗集当且仅当\(\overline{A}\)是疏朗集.这就等价于\(\overline{A}^c=X/\overline{A}\)在\(X\)中稠密.
	\end{definition}
	\begin{note}
		\begin{enumerate}
			\item 稠密性的定义:
			
			\(A\)在\(X\)中稠密\(\Leftrightarrow\)\(X \subset \overline{A}\)\(\Leftrightarrow\)任意\(X\)中一点\(x_0\),存在\(x_0\)的一个邻域\(V_0\)使得\(V_0\cap A \ne \varnothing.\)
			\item 由定义看出,一个集合是不是疏朗集和这个集合上采取的拓扑有关.
		\end{enumerate}
	\end{note}
	有了疏朗集的定义,我们就能描述子集中元素的多少情况了.
	\begin{definition}[纲]
		度量空间\(X\)中,如果子集\(A\)可以写成可数个疏朗集的并,则称\(A\)是第一纲的,否则称\(A\)为第二纲的.
	\end{definition}
	\begin{note}
		设\(A_i\)是第一纲集族,则\(\bigcup_{i=0}^\infty A_i\)也是第一纲的.
	\end{note}
	\begin{example}
		在\((\mathbb{R},||\cdot ||)\)中,单点集\(\{q\}\)是疏朗集,因此整数集\(\mathbb{Z}\)是第一纲的;然而在\((\mathbb{Z},||\cdot ||)\)这个平凡的度量空间中,由于每一个点\(\{p\}\)都是开集,因此不是疏朗集.故此在这个拓扑下\(\mathbb{Z}\)是第二纲的.
	\end{example}
	
	\begin{theorem}[强Baire纲定理(3.1.2)]
		设度量空寂\((X,d)\)完备,若\(\{U_n\}\)是\(X\)中一族稠密开集,则\(\bigcap_n U_n\)是第二纲的稠密集.
	\end{theorem}
	\begin{theorem}[Baire(3.1.1)]
		完备度量空间是第二纲的.
	\end{theorem}
	\begin{proof}
		反证法:若完备度量空间\(X\)是第一纲的,则存在疏朗集列\(A_i\),使得\(X=\bigcup_i A_i\).因此\(X/\bigcup_i A_i=\bigcap_i A_i^c\),其中\(A_i^c\)是稠密开集,因此\(X/\bigcup_i A_i\ne \varnothing\),这和\(X\)是第一纲的矛盾.
	\end{proof}
	\begin{example}
		无理数\(\mathbb{R}/\mathbb{Q}\)是第二纲的.若不然,则它是第一纲的,又由于\(\mathbb{Q}\)是第一纲的,因此\(\mathbb{R}/\mathbb{Q}\cup \mathbb{Q}=\mathbb{R}\)是第一纲的,这与\(\mathbb{R}\)的完备性矛盾.
	\end{example}
	\begin{example}
		见例3.1.4和3.1.5,存在Lebesgue测度为0的第二纲集,也存在Lebesgue测度大于0的第一纲集.
	\end{example}
	
	\section{开映射定理,逆算子定理,闭图像定理,共鸣定理}
	\subsection{开映射定理、逆算子定理}
	\begin{definition}
		设\(X,Y\)是两个赋范线性空间,\(T\in \mathfrak{B}(X,Y)\).
		\begin{enumerate}
			\item 若\(T\)是一对一的,即\(x_1\ne x_2\)蕴含\(Tx_1\ne Tx_2\),则称\(T\)是单射.
			\item 设\(T\)的值域\(\mathfrak{R}(T)=Y\),则称\(T\)是满射,或者\(T\)是到上的.
			\item 若\(T\)既是单射又是满射,且\(T^{-1}\in  \mathfrak{B}(X,Y)\),则称\(T\)是可逆的.
		\end{enumerate}
	\end{definition}
	\begin{note}
		\begin{enumerate}
			\item \(T\)是可逆的,当且仅当存在\(J\in \mathfrak{B}(Y,X)\)使得\(TJ=I_Y,\ JT=I_X\).
			\item 有限维空间中,左逆或右逆存在即可推出逆算子存在,但是无限维空间就不行.
		\end{enumerate}
	\end{note}
	设\(X,Y\)是两个赋范线性空间,\(T\in \mathfrak{B}(X,Y)\),且\(T\)是一一对应,则\(T^{-1}\)存在且也是线性算子,但是不一定有界(如求导算子).然而,逆算子定理给出了\(T^{-1}\)有界的条件.
	
	连续算子的像若是开集,则原像也是开集.与其相对应,把开集映为开集的映射是开映射.注意开映射和连续算子是不同的概念.开映射的例子有如商映射,全纯函数等.
	\begin{theorem}[开映射定理(3.2.1)]
		设\(X,Y\)是两个Banach空间,\(T\in \mathfrak{B}(X,Y)\).若\(TX=Y\),则\(T\)是开映射.
	\end{theorem}
	利用开映射定理,我们可以得到逆算子定理.
	\begin{theorem}[逆算子定理(3.2.2)]
		设\(X,Y\)是两个Banach空间,\(T\in \mathfrak{B}(X,Y)\),且\(T\)是一一对应,则逆算子\(T^{-1}\)有界.
	\end{theorem}
	\begin{proof}
		由开映射定理,\(T\)是开映射.任取\(X\)中开集\(U\),\(TU=(T^{-1})^{-1}U\)是开集,因此\(T^{-1}\)连续.
	\end{proof}
	下面是逆算子定理在范数等价方面的一个应用.
	\begin{theorem}[范数等价定理]
		设线性空间\(X\)对其上定义的范数\(||\cdot ||_1,\ ||\cdot||_2,\ \)都成为Banach空间,且存在正数\(c\)使得
		\[||x||_1\le c||x||_2,\ \forall x \in X,\]
		则\(||\cdot ||_1,\ ||\cdot||_2\)等价.
	\end{theorem}
	\begin{proof}
		把恒等算子\(I\)看作\((X,||\cdot||_1)\to (X,||\cdot||_2)\)的线性算子,则
		\[||Ix||_1=||x||_1\le c||x||_2,\]
		即\(I\)有界,又\(I\)是一一对应,因此由逆算子定理,\(I^{-1}=I\)有界,因此存在正数\(c'\)满足
		\[||x||_2=||I^{-1}x||_2\le c'||x||_1.\]
		即\(||\cdot ||_1\) 弱于\(||\cdot||_2\),因此两个范数等价.
	\end{proof}
	
	\subsection{闭图像定理}
	先定义映射的图像.
	\begin{definition}[映射的图像]
		设有集合\(X,Y\),映射\(T: \mathfrak{D}(T)\to Y\).做\(X,Y\)的乘积集\(X\times Y=\{(x,y):x\in X,\ y \in Y\}\),称\(X\times Y\)的子集
		\[\mathfrak{G}(T)=\{(x,Tx):x\in \mathfrak{D}(T)\}\]
		为映射\(T\)的图像.
	\end{definition}
	引进\(X,Y\)中的度量\(d_X,\ d_Y\),使其称为度量空间,则\((X\times Y,\ d)\)也是度量空间,其中\(d((x_1,y_1),(x_2,y_2))=\sqrt{d_X(x_1,x_2)^2+d_Y(y_1,y_2)^2}.\)
	\begin{definition}[闭算子]
		设\((X,d_X),(Y,d_Y)\)是两个度量空间,\(T:\mathfrak{D}(T)\to Y\),若\(T\)的图像\(\mathfrak{G}(T)\)是\((X\times Y,d)\)中的闭集,则称\(T\)是闭算子.
	\end{definition}
	闭算子的判定充要条件如下:
	\begin{lemma}
		设有度量空间\(X,Y\)和映射\(T:\mathfrak{D}(T)\to Y\),则\(T\)是闭算子的充要条件是,对任何\(\mathfrak{D}(T)\)中点列\(\{x_n\}\),当\(x_n\to x_0,\ y_n=Tx_n\to y_0\)时,就有\(x_0\in \mathfrak{D}(T)\),且\(y_0=Tx_0\).
	\end{lemma}
	\begin{lemma}
		定义域是闭集的连续算子是闭算子.但是闭算子不一定是连续算子.
	\end{lemma}
	\begin{theorem}[闭图像定理(3.2.5)]
		设\(X,Y\)是两个Banach空间,\(T:\mathfrak{D}(T)\to Y\)是闭线性算子,若\(\mathfrak{D}(T)\)是\(X\)中的闭线性子空间,则\(T\)连续.
	\end{theorem}
	\begin{example}
		设\(f\in \mathcal{M}[0,1]\),若对于任意的\(g\in L^2[0,1]\)都有\(fg\in L^1[0,1]\),则\(f\in L^2[0,1]\).
	\end{example}
	\begin{proof}
		定义乘法算子\(T(g)=fg\),则\(T\)是\(L^2[0,1]\to L^1[0,1]\)的闭算子.由闭图像定理,由于\(L^2[0,1]\)是闭线性子空间,因此\(T\)连续,即\(T\)有界.取\(g_n=f\chi_{|f|<n}\in L^2[0,1]\),则\(||g_n||_2=\frac{||fg_n||_1}{||g_n||_2}\le ||T||\),再令\(n\to \infty\),就得到\(||f||_2\le||T||,f\in L^2[0,1]\).
		
		下面证明\(T\)是闭算子.对于\(g_n\xrightarrow{L^2}g,\ fg_n\xrightarrow{L^1}h\),由Chebyshev知道函数列\(\{g_n\},\{fg_n\}\)分别依测度收敛于\(g,\ h\).再由Riesz,存在子列\(\{g_{n_k}\},\{fg_{n_k}\}\)几乎处处收敛于\(g,\ h\).于是知道当\(g_n\to g\),就有\(Tg_n=fg_n\to h=fg=Tg,\ g\in L^2[0,1]\).即\(T\)是闭算子.
	\end{proof}
	\subsection{共鸣定理}
	\begin{theorem}[共鸣定理(一致有界原理)(3.2.6)]
		设\(X\)是Banach空间,\(Y\)是赋范线性空间,\(\{T_\lambda:\lambda\in \Lambda\}\subset \mathfrak{B}(X,Y)\),若对于任意的\(x\in X\),都有
		\[\sup_{\lambda\in \Lambda}||T_\lambda x||<\infty,\]
		则数集\(\{||T_\lambda||:\lambda\in \Lambda\}\)有界.
		
		用Baire纲定理,可以给出共鸣定理的另一种形式:
		
		设赋范线性空间\(X,Y\),\(\{T_\lambda:\lambda\in \Lambda\}\subset \mathfrak{B}(X,Y)\).若
		\[\{x\in X:\sup_{\lambda\in \Lambda}||T_\lambda x||<\infty\}\]
		是\(X\)的第二纲集,则数集\(\{||T_\lambda||:\lambda\in \Lambda\}\)有界.
	\end{theorem}
	\begin{example}
		设\(p\ge 1,\alpha(t)\)是实直线\(\mathbb{R}\)中闭区间\([a,b]\)上的可测函数.若对任意\(x\in L^p[a,b]\),积分
		\[\int_{a}^{b}\alpha(t)x(t)dt\]
		存在,则\(\alpha(t)\in L^q[a,b],\ \frac{1}{p}+\frac{1}{q}=1.\)
	\end{example}
	\begin{proof}
		取\(x(t)\equiv 1\),则\(\alpha(t)\in L^1[a,b]\).因此\(\alpha\)几乎处处有限.作函数列\(\alpha_n=\alpha\chi_{|\alpha|<n\),则\(\alpha_n\)几乎处处收敛于\(\alpha.\)作\(L^p[a,b]\)上的连续线性泛函
			\[F_n(x)=\int_{a}^{b}x(t)\alpha_n(t)dt,\ \forall x\in L^p[a,b],\]
			由于\(|x(t)\alpha_n(t)|\le |x(t)\alpha(t)|\in L^1[a,b]\),由Lebesgue控制收敛定理:
			\[\lim_{n\to \infty}F_n(x)=\int_{a}^{b}\alpha(t)x(t)dt,\ \forall x\in L^p[a,b].\]
			即\(\sup_n |F_n(x)|< \infty.\)
			由共鸣定理:存在数\(M\)使得\(||F_n||=||\alpha_n||_q<M\),令\(n\to \infty\),知道\(||\alpha||_q<M.\)即\(\alpha \in L^q[a,b]\).
	\end{proof}
	
	
	\chapter{常用的度量空间符号及范数}
	\begin{enumerate}
		\item \(\mathbb{R}^n\):数列全体
		\item \(l^\infty\):有界数列全体. \(||x||_\infty=sup_{n\in \mathbb{N}}|x_n|\)
		\item \(c\):收敛数列全体
		\item \(c_o\):收敛至0的数列全体
		\item \(l^p=\{x:x=(x_1,x_2,...),\sum_{n\ge 1}|x_n|^p<\infty\}(1 \le p< \infty),\)
		
		\(||x||_p=(\sum_{n \ge 1}|x_n|^p)^{1/p}\).
		\item \(C[a,b]\):在区间\([a,b]\)连续的函数全体 \(||f||_c=max_{a\le x\le b}|f(x)|\)
		\item \(L^p(E)=\{f:f\in \mathscr{M}(E),||f||_p=(\int_E|f|^p)^{1/p}<\infty\}\).
		
		即在Legesgue可测集\(E\)上\(p\)次Lebesgue可积的函数类.
		\item \(L^\infty(E)=\{f:f\in \mathscr{M}(E),\exists M\in \mathbb{R},s.t.m\{|f|>M\}=0\}.||f||_\infty = inf\{M:|f|\le M,x\in E\}.\)
		\item \(P(x)\):一元多项式全体
		\item \(C^\infty(U)\):\(\mathbb{R}^n\)中开集\(U\)上的无限次可微函数全体
		\item \(C_c(U)\):\(U\)上紧支撑的连续函数全体
	\end{enumerate}
	
\end{document}