\documentclass[lang=cn,10pt]{elegantbook}

\title{统计学基础}


\extrainfo{咕咕嘎嘎}

\setcounter{tocdepth}{3}


\usepackage{pdfpages}
\usepackage{array}
\usepackage{tikz-cd}
\usepackage{amsmath}
\usepackage{amssymb}
\usepackage{braket}
\usepackage{faktor}
\addbibresource{refs.bib} % 指定 .bib 文件(含扩展名)
% 修改标题页的橙色带
\definecolor{customcolor}{RGB}{30, 60, 114}
\colorlet{coverlinecolor}{customcolor}
\cover{steve-busch-Zm0MOhF7iHE-unsplash (1).pdf}
\newcommand{\ccr}[1]{\makecell{{\color{#1}\rule{1cm}{1cm}}}}

\begin{document}
	
	\definecolor{structurecolor}{RGB}{18, 33, 79}
	\definecolor{main}{RGB}{88, 63, 125}
	\definecolor{second}{RGB}{220, 220, 235}
	\definecolor{third}{RGB}{46, 139, 139}
	
	\maketitle
	\frontmatter
	
	\tableofcontents
	
	\mainmatter
    \chapter{数理统计基础}

    \chapter{线性模型}
    \begin{introduction}
    \item 一元线性模型
    \item 多元线性模型
    \end{introduction}
    我们希望从数据上找到两个现象之间的关系,比如说物理学中我们寻找质量不变时加速度和力的关系到底是怎样的?我们应该用什么函数模型来描述现象之间的关系?为了解决这些问题,我们把得到的数据画到坐标纸上,试图用一条线来经过尽可能多的点——我们称这种行为为“拟合”。这里就蕴含了线性回归的思想:尝试用一个“回归方程”来靠近尽可能多的数据点,并使这个方程与所有点的“距离”尽可能地小,从而得到一个数学式的解释。我们在线性模型中就要解决这种问题:我们要建立什么样的模型,在什么样的标准下才能最小化“距离”之和?这个模型和它的参数又有什么性质?这个模型真实可靠、有现实意义吗?下面我们从一元线性模型开始阐述这个课题。
    \section{一元线性模型}
    一元回归分析研究一个变量(因变量\(y\))对一个解释变量(自变量\(x\))的依从关系。
    \subsection{回归直线}

    我们假设,在建模合理的情况下(变量种类、数量选择合理,它们的关系也处理得当)数据在回归直线旁两侧随机波动。也就是说:如果我们设回归直线为\(f(x)=E[Y|X]\)的话,那么
    \[y=E[Y|X]+\mu.\]
    这里的条件期望我们称为回归函数,它是在已知\(x\)时对\(y\)在均方误差意义下最好的估计量。关于条件期望的详细内容,见An\&P条件期望部分内容。需要注意的是,这里的\(x\)虽然被称为自变量,但是我们把它看作固定的常数。事实上,它们就是我们收集到的数据,自然是已知的。而这里随机性的来源是\(\mu\),它是一个随机变量,因此\(y\)也是一个随机变量。\(\mu\)就是数据在回归直线两侧的波动,我们称其为离差,或随机误差项。它是无法观测的。离差代表了样本点和回归方程的距离。

    在线性回归中,我们取回归函数\(E[Y|X]\)为线性函数——这里的“线性”是相对于系数\(\beta_i\)而非自变量\(x_i\)而言的。一元线性回归中,我们取回归函数
    \[E[Y|X]=\beta_0+\beta_1x.\]
    于是模型就变为
    \[y_i=\beta_0+\beta_1x_i+\mu_i.\]
    我们称这个式子为总体回归函数,我们在散点图中画的拟合直线并不是总体回归函数,因为总体回归函数总是未知的。注意,这里的\(\beta_0,\beta_1\)都是未知但是固定的数,我们要估计这两个参数。

    所谓的“拟合”就是通过估计模型的参数试图预测两个现象之间的数量关系。我们手上拿着两种数据:自变量\(x_i\)和因变量\(y_i\),现在就要通过估计的模型算出因变量的估计\(\hat{y_i}\),这要通过估计模型中的两个参数\(\beta_0,\beta_1\)来实现,这时
    \[\hat{y_i}=\hat{\beta_0}+\hat{\beta_1}x_i.\]
    这就是样本回归函数。我们用样本回归函数估计总体回归函数。

    在试图用样本回归函数估计了总体回归函数后,我们当然不能期望两者完全一样。也就是存在残差\(\hat{\mu_i}\):
    \[y_i=\hat{y_i}+\hat{\mu_i}.\]
    以上就是用样本回归函数估计总体回归函数的想法。我们现在更细致一些来讨论线性回归模型的假设和性质。
    \subsection{最小二乘估计}

    经典线性回归模型(CLRM)对随机项\(\mu\)做了以下假设,这些假设使最小二乘估计得出的参数估计具有优良性质:
    \begin{enumerate}
        \item 条件均值为0:\[E[\mu_i|X_i]=0.\]
        \item 同方差:\[V[\mu_i|X_i]=\sigma_\mu^2.\]
        \item 无序列相关:\[cov(\mu_i,\mu_j|X_i,X_j)=0,i\ne j.\]
        \item 正态假设:\[\mu_i\sim N(0,\sigma_\mu^2).\]
    \end{enumerate}

    如何估计模型中的参数\(\beta_0,\beta_1\)?我们想画出来的拟合线当然是越靠近数据点越好。但是怎么定义“近”这个概念?自然联想到距离。但是欧氏距离在计算上不好处理,因此用估计直线和样本点之间的距离平方和,也就是残差平方和
    \[Q=\sum_{i=1}^{n}\hat{\mu_i}^2=\sum_{i=1}^{n}(y_i-\hat{\beta_0}-\hat{\beta_1}x_i)^2\]
    来衡量。我们要最小化\(Q\),这只要求导求出极值点即可。
    \begin{proof}
        分别对\({\beta_0},{\beta_1}\)求偏导:
        \begin{align*}
        \left.\frac{\partial Q}{\partial\beta_0}\right|_{\beta_0=\hat{\beta}_0}
        &= -2\sum_{i=1}^{n}\bigl(y_i-\hat{\beta}_0-\hat{\beta}_1 x_i\bigr) = 0,\\
        \left.\frac{\partial Q}{\partial\beta_1}\right|_{\beta_1=\hat{\beta}_1}
        &= -2\sum_{i=1}^{n}\bigl(y_i-\hat{\beta}_0-\hat{\beta}_1 x_i\bigr)x_i = 0.
        \end{align*}
        整理上面的方程:
        \begin{align*}
            \hat{\beta}_0+\hat{\beta}_1\overline{x}&=\overline{y},\\
            \hat{\beta}_0(\sum_{i=1}^{n}x_i)+\hat{\beta}_1(\sum_{i=1}^{n}x_i^2)&=\sum_{i=1}^{n}x_iy_i.
        \end{align*}
        于是得到
        \begin{align*}
            \hat{\beta}_0&=\overline{y}-\hat{\beta}_1\overline{x},\\
            \hat{\beta}_1&=\frac{L_{xy}}{L_{xx}}.
        \end{align*}
        其中
        \[L_{xy}=\sum_{i=1}^{n}(x_i-\overline{x})(y_i-\overline{y})=\sum_{i=1}^{n}x_iy_i-n\overline{x}\overline{y},\]
        \[L_{xx}=\sum_{i=1}^{n}(x_i-\overline{x})^2=\sum_{i=1}^{n}x_i^2-n\overline{x}^2.\]
    \end{proof}
    \begin{note}
    实际上求偏导直接得出的方程组说明了残差的两条性质:
    \begin{enumerate}
        \item \(\sum_{i=1}^{n}\mu_i=0\);
        \item \(\sum_{i=1}^{n}\mu_ix_i=0\).
    \end{enumerate}
    \end{note}

    \subsection{最小二乘估计的性质}
    在经典线性回归模型的假设下,最小二乘估计具有很好的性质。
    \begin{property}
        \begin{enumerate}
            \item 线性性:\(\hat{\beta_0},\hat{\beta_1}\)都是\(y_i\)的线性函数,于是\(\hat{\beta_0},\hat{\beta_1}\)都服从正态分布;
            \item 无偏性:\(E[\hat{\beta_0}]=\beta_0,E[\hat{\beta_1}]=\beta_1\);
            \item 有效性(Gauss-Markov):它是所有线性无偏估计量中具有最小方差的估计;
        \end{enumerate}
    \end{property}
    \begin{proof}
        我们证前两条,并顺带得出两个参数的方差,进而得到两个参数的分布。
        \begin{enumerate}
            \item 首先,可以证明\(\hat{\beta_1}\)可以改写成\[\frac{\sum_{i=1}^n(x_i-\overline{x})y_i}{\sum_{j=1}^{n}(x_j-\overline{x})^2},\]这就把\(\hat{\beta_1}\)改写成了\(y_i\)的线性组合。同理我们能把\(\hat{\beta_0}\)写成\[\frac{\sum_{i=1}^n[\frac{L_{xx}}{n}-\overline{x}(x_i-\overline{x})]y_i}{L_{xx}}.\]
            \item 现在证明无偏性。我们知道\(y_i\)也服从正态分布\(N(\beta_0+\beta_1x_i,\sigma^2)\),于是由期望线性性就知道\[E[\hat{\beta_1}]=\frac{\sum_{i=1}^n(x_i-\overline{x})}{\sum_{j=1}^{n}(x_j-\overline{x})^2}E[y_i]=\frac{\sum_{i=1}^n(x_i-\overline{x})}{\sum_{j=1}^{n}(x_j-\overline{x})^2}(\beta_0+\beta_1x_i)=\beta_1.\]其中用了\(\sum_{i=1}^n(x_i-\overline{x})=0,\sum_{i=1}^n(x_i-\overline{x})x_i=\sum_{i=1}^n(x_i-\overline{x})^2.\)
            
            于是\(\beta_0\)就有:\[E[\beta_0]=E[\overline{y}-\hat{\beta}_1\overline{x}]=\beta_0+\beta_1x_i-\beta_1 x_i=\beta_0.\]
            
            下面我们来看两个参数的方差。由于\(\mu_i\)互相独立,因此也有\(y_i\)互相独立。于是\[V[\hat{\beta_1}]=\sum_{i=1}^{n}(\frac{x_i-\overline{x}}{L_{xx}})^2V[y_i]=\frac{\sigma^2}{L_{xx}}.\]
            因此\(\hat{\beta_1}\sim N(\beta_1,\frac{\sigma^2}{L_{xx}})\).

            另外,\[V[\hat{\beta_0}]=\sigma^2\sum_{i=1}^{n}(\frac{1}{n}-\frac{\overline{x}(x_i-\overline{x})}{L_{xx}})^2=(\frac{1}{n}+\frac{\overline{x}^2}{L_{xx}})\sigma^2.\]
            因此\(\hat{\beta_0}\sim N(\beta_0,(\frac{1}{n}+\frac{\overline{x}^2}{L_{xx}})\sigma^2)\).
        \end{enumerate}
    \end{proof}
    \begin{note}
        我们从上面得出的参数分布可知,对于不同于\(x_i,i=1,2,\cdots n\)的样本点\(x_0\)来说,它的预测值\(\hat{y}_0=\hat{\beta}_0+\hat{\beta}_1x_0\)同样服从正态分布,即
        \[\hat{y}_0\sim N(\beta_0+\beta_1x_0,(\frac{1}{n}+\frac{(x_0^2-\overline{x})^2}{L_{xx}})\sigma^2).\]
    \end{note}

        我们得出的样本回归线有几条性质:
    \begin{enumerate}
        \item 过点\((\overline{x},\overline{y})\);
        \item \(E[\hat{y_i}]=E[y_i]\);
        \item 残差估计值和\(\hat{y_i},x_i\)都不相关。
    \end{enumerate}

    \subsection{极大似然估计}
    从极大似然估计的角度,我们也可以得出\(\hat{\beta}_0,\hat{\beta}_1\),还有\(\sigma^2\)的估计。首先写出似然函数,由\(y_i\sim N(\beta_0+\beta_1 x_i,\sigma^2)\):
    \[f(y;x_1,x_2,\cdots x_n)=\frac{1}{(\sqrt{2\pi}\sigma)^n}\exp{(-\frac{\sum_{i=1}^{n}(y-\beta_0-\beta_1 x_i)^2}{2\sigma^2})},\]
    取对数:
    \[\log{f}=-\frac{n}{2}\log{(2\pi\sigma^2)}-\frac{\sum_{i=1}^{n}(y-\beta_0-\beta_1 x_i)^2}{2\sigma^2},\]
    对于\(\hat{\beta}_0,\hat{\beta}_1\),求最值的步骤和最小二乘法相同。下面我们看\(\sigma^2\)的极大似然估计。对\(\sigma^2\)求偏导并令其为零:
    \[\frac{\partial}{\partial \sigma^2}\log{f}=-\frac{n}{2\sigma^2}+\frac{\sum_{i=1}^{n}(y-\beta_0-\beta_1 x_i)^2}{2(\sigma^2)^2}=0,\]
    解得
    \[\hat{\sigma}^2=\frac{1}{n}\sum_{i=1}^{n}\hat{\mu_i}^2.\]

    然而\(\hat{\sigma}^2\)并不是\(\sigma^2\)的无偏估计。我们经常使用的是无偏估计版本:
    \[\hat{\sigma}^2=\frac{1}{n-2}\sum_{i=1}^{n}\hat{\mu_i}^2.\]

    
    \subsection{一元线性回归模型的检验}



    \chapter{时间序列分析}

    \chapter{统计模拟}

    \chapter{抽样方法}



\end{document}