\documentclass[lang=cn,10pt]{elegantbook}

\title{复变函数速成}


\extrainfo{咕咕嘎嘎}

\setcounter{tocdepth}{3}

\usepackage{array}
\newcommand{\ccr}[1]{\makecell{{\color{#1}\rule{1cm}{1cm}}}}


\usepackage{pdfpages}
\cover{katyayan-gauniyal-1c0rfpjBc_k-unsplash (1).pdf}
\usepackage{amssymb}
\usepackage{braket}
% 修改标题页的橙色带
\definecolor{customcolor}{RGB}{17,97,98}
\colorlet{coverlinecolor}{customcolor}

\begin{document}
	\definecolor{structurecolor}{RGB}{15, 5, 128}
	\definecolor{main}{RGB}{42, 108, 145}
	\definecolor{second}{RGB}{119, 164, 189}
	\definecolor{third}{RGB}{82, 191, 182}
	
	\maketitle
	\frontmatter
	
	\tableofcontents
	
	\mainmatter


		本手册浓缩了复积分(留数法)中常用的工具、Jordan 大弧与小弧引理、以及若干典型例题的解法步骤.
		
	\chapter{全纯函数}
	\section{Cauchy--Riemann 方程}
	
	\begin{theorem}[解析(全纯)函数的定义]
		设 \(U\subset\mathbb C\) 为开集。函数 \(f:U\to\mathbb C\) 在 \(z_0\in U\) 处称为全纯(解析),若复导数
		\[
		f'(z_0)=\lim_{h\to0}\frac{f(z_0+h)-f(z_0)}{h}
		\]
		存在;若对 \(U\) 中每一点均存在该极限,则称 \(f\) 在 \(U\) 全纯。
	\end{theorem}
	
	\begin{definition}[Cauchy--Riemann 方程]
		设 \(f=u+iv\)(\(u,v\) 为实值函数),若 \(u,v\) 在点 \(z_0=(x_0,y_0)\) 的偏导存在且连续,则 \(f\) 在该点可复微的充分必要条件为
		\[
		u_x=v_y,\qquad u_y=-v_x
		\]
		(在点处成立)。
	\end{definition}
	
	\begin{proof}
		(要点)从复定义出发,写
		\[
		\frac{f(z_0+h)-f(z_0)}{h}=\frac{u(x_0+\Delta x,y_0+\Delta y)-u(x_0,y_0)}{\Delta x+i\Delta y}
		+i\frac{v(x_0+\Delta x,y_0+\Delta y)-v(x_0,y_0)}{\Delta x+i\Delta y}.
		\]
		分别沿实轴方向与虚轴方向取极限,并比较实部与虚部,得到 CR 方程;带反向条件并用偏导连续性可将局部实可微性与 CR 推回复可微性。
	\end{proof}
	
	\begin{example}
		令 \(f(z)=\overline z\). 写 \(u(x,y)=x,\ v(x,y)=-y\)。代入 CR 得 \(u_x=1\), \(v_y=-1\),不满足 \(u_x=v_y\),因此 \(f\) 处处不可复微(除非退化情形)。
	\end{example}
	
	\section{Cauchy 积分定理与 Cauchy 积分公式}
	
	\begin{definition}[Cauchy 积分定理]
		设 \(U\) 是单连通开集,\(f\) 在 \(U\) 全纯。若 \(\Gamma\subset U\) 为分段光滑的闭曲线,则
		\[
		\oint_\Gamma f(z)\,dz=0.
		\]
	\end{definition}
	
	\begin{proof}
		(要点)可用 Green 定理把复积分转为实二重积分:写 \(f=u+iv\),则
		\[
		\oint_\Gamma f(z)\,dz = \int\!\!\int_D\left(\frac{\partial v}{\partial x}+\frac{\partial u}{\partial y}\right)\,dx\,dy + i\int\!\!\int_D\left(\frac{\partial u}{\partial x}-\frac{\partial v}{\partial y}\right)\,dx\,dy,
		\]
		CR 方程使被积为零,从而积分为 0。另一种更抽象的证明用同伦变形或原始函数存在性。
	\end{proof}
	
	\begin{definition}[Cauchy 积分公式]
		设 \(f\) 在闭盘 \(\overline{D(a,R)}\) 上解析,则对任意 \(|z-a|<R\),
		\[
		f(z)=\frac{1}{2\pi i}\oint_{|w-a|=R}\frac{f(w)}{w-z}\,dw,
		\]
		并且对任意整数 \(n\ge0\),
		\[
		f^{(n)}(z)=\frac{n!}{2\pi i}\oint_{|w-a|=R}\frac{f(w)}{(w-z)^{n+1}}\,dw.
		\]
	\end{definition}
	
	\begin{proof}
		(要点)对固定 \(z\) 把 \(\dfrac{f(w)}{w-z}\) 看作 \(w\) 的全纯函数(在去掉 \(w=z\) 的区域),用留数定理在小圆上计算积分,或将 \(f(w)\) 在 \(w\) 处展开为 Taylor 级数并逐项积分,得出公式。
	\end{proof}
	
	\begin{example}
		由 Cauchy 公式可推导 Cauchy 导数估计:若 \(|f(w)|\le M\) 在 \(|w-a|=R\) 上成立,则
		\[
		|f^{(n)}(a)|\le \frac{n! M}{R^n}.
		\]
		这在许多处证明唯一延拓、Liouville 定理时非常有用。
	\end{example}
	
	\section{Taylor 与 Laurent 展开}
	
	\begin{definition}[Taylor 展开(局部)]
		若 \(f\) 在 \(D(a,R)\) 全纯,则存在唯一系数列 \(\{c_n\}\) 使
		\[
		f(z)=\sum_{n=0}^\infty c_n (z-a)^n,\qquad |z-a|<R,
		\]
		且 \(c_n=f^{(n)}(a)/n!\)。
	\end{definition}
	
	\begin{proof}
		(要点)由 Cauchy 导数公式可写
		\[
		c_n=\frac{1}{2\pi i}\oint_{|w-a|=\rho}\frac{f(w)}{(w-a)^{n+1}}\,dw,\quad 0<\rho<R,
		\]
		由此得到幂级数展开与收敛半径。
	\end{proof}
	
	\begin{definition}[Laurent 展开]
		若 \(f\) 在环域 \(A=\{z: r<|z-a|<R\}\) 全纯,则存在唯一 Laurent 级数
		\[
		f(z)=\sum_{n=-\infty}^\infty c_n (z-a)^n
		\]
		在该环域内一致收敛。Laurent 展开中的负幂部分描述了 \(f\) 在中心点的奇性。
	\end{definition}
	
	\begin{example}
		函数 \(\dfrac{1}{z(z-1)}\) 在 \(0<|z|<1\) 的 Laurent 展开可由部分分式分解:
		\[
		\frac{1}{z(z-1)}=-\frac{1}{z} - \sum_{n=0}^\infty z^n,\qquad 0<|z|<1,
		\]
		可见 \(z=0\) 为简单极点。
	\end{example}
	
	\section{留数与实积分}
	
	\begin{definition}[留数定理]
		设 \(U\) 为单连通开集,\(f\) 在 \(U\) 上解析,除有限孤立奇点 \(a_1,\dots,a_n\)。若 \(\Gamma\subset U\) 为逆时针简单闭曲线将这些奇点包住,则
		\[
		\oint_\Gamma f(z)\,dz = 2\pi i\sum_{j=1}^n \operatorname{Res}(f,a_j).
		\]
	\end{definition}
	
	\begin{proof}
		(要点)取每个奇点小圆 \(C_j\)(互不相交)并把 \(\Gamma\) 通过形变替换为这些小圆的合(方向相反),再对每个小圆用 Laurent 展开逐项积分,只留下 \(c_{-1}\) 项,从而得到结论。
	\end{proof}
	
	
	\section{奇点的分类(可去 / 极点 / 本性)}
	
	\begin{theorem}[孤立奇点的分类(定义)]
		设 \(a\) 为孤立奇点。按 Laurent 展开中的负次项个数分类:
		\begin{itemize}
			\item 若 Laurent 负次项全为 0,则为 \emph{可去奇点};
			\item 若 Laurent 仅有有限多项负次项(最大为 \((z-a)^{-m}\)),则为 \emph{极点}(阶为 \(m\));
			\item 若 Laurent 具有无限多负次项,则为 \emph{本性奇点}。
		\end{itemize}
	\end{theorem}
	
	\begin{definition}[Casorati--Weierstrass 与 Picard 小定理]
		若 \(a\) 为本性奇点,则 \(f\) 在任何去心邻域内的像在 \(\mathbb C\) 中稠密(Casorati--Weierstrass)。Picard 小定理更强:在本性奇点任意邻域内,\(f\) 取得所有复值,至多舍去一个例外值。
	\end{definition}
	
	\begin{example}
		\(e^{1/z}\) 在 \(z=0\) 处为本性奇点;在任意小环域中其值几乎取遍整个复平面(除 0 外也能任意接近 0)。
	\end{example}
	
	\section{最大模原理与 Liouville 定理}
	
	\begin{definition}[最大模原理]
		若 \(f\) 在连通域 \(U\) 全纯且存在 \(z_0\in U\) 使得 \(|f(z_0)|\ge |f(z)|\) 对所有 \(z\in U\) 成立,则 \(f\) 为常数。
	\end{definition}
	
	\begin{definition}[Liouville 定理]
		若 \(f\) 在全平面 \(\mathbb C\) 上全纯且有界,则 \(f\) 为常数。
	\end{definition}
	
	\begin{proof}
		(Liouville 简洁证法)设 \(|f(z)|\le M\) 对所有 \(z\) 成立。对任意 \(R>0\),Cauchy 导数估计给出
		\[
		|f'(0)|\le \frac{M}{R}.
		\]
		令 \(R\to\infty\) 得 \(f'(0)=0\)。把中心移到任意点可得 \(f'\equiv0\),从而 \(f\) 常数。
	\end{proof}
	
	\begin{example}
		用 Liouville 可证明基本代数定理:若多项式 \(p(z)\) 无根,则 \(1/p(z)\) 在全平面解析且有界(因 \(|p(z)|\to\infty\) 当 \(|z|\to\infty\)),由 Liouville 得常数矛盾,因此 \(p\) 有根。
	\end{example}
	
	\section{解析延拓(Analytic continuation)}
	
	\begin{theorem}[局部定义与全纯性的回顾]
		设 \(U\subset\mathbb C\) 为开集,\(f:U\to\mathbb C\) 在 \(U\) 全纯。若 \(a\in U\),则
		存在以 \(a\) 为中心的圆盘 \(D(a,r)\subset U\) 以及幂级数
		\[
		f(z)=\sum_{n=0}^\infty c_n (z-a)^n,\qquad |z-a|<r,
		\]
		在该圆盘上一致收敛。这给出“局部数据”去构造更大域上的函数的出发点。
	\end{theorem}
	
		\begin{definition}[恒等定理(Identity Theorem)]
		若 \(f\) 与 \(g\) 在连通域 \(U\) 上全纯,且 \(f(z)=g(z)\) 在 \(U\) 的一集合上成立,该集合有聚点,则 \(f\equiv g\) 在 \(U\) 上恒等。
	\end{definition}
	
	\begin{proof}
		(恒等定理要点)令 \(h=f-g\),则 \(h\) 在 \(U\) 全纯,且在有聚点的集合上为 0。若 \(h\not\equiv0\),则零点必须是孤立的,与有聚点矛盾,故 \(h\equiv0\)。
	\end{proof}
	
	\begin{example}
		幂级数 \(\sum_{n=0}^\infty z^{n!}\) 在单位圆内解析,若其在某有聚点的集合上与 0 相等,则恒为 0(由恒等定理)。
	\end{example}
	
	\begin{definition}[解析延拓(analytic continuation)——直观定义]
		设 \(f\) 在开集 \(U\) 全纯,\(V\) 是包含 \(U\) 的更大开集。若存在在 \(V\) 全纯的函数 \(F\) 使得 \(F|_U=f\),则称 \(F\) 是 \(f\) 在 \(V\) 上的解析延拓。解析延拓(若存在)在交集中有聚点时是唯一的(由恒等定理)。
	\end{definition}
	
	\begin{definition}[延拓的基本原理(Principle of analytic continuation)]
		若 \(f\) 在连通开集 \(U\) 全纯,且存在开集链
		\[
		U=U_0\subset U_1\subset\cdots\subset U_N
		\]
		使得对每个 \(k\) 存在全纯函数 \(f_k\) 在 \(U_k\) 上,且 \(f_k|_{U_{k-1}}=f_{k-1}\),则把这些局部定义粘接得到 \(f_N\) 即为沿这条链的解析延拓。换句话说:从幂级数或积分表示“向外延伸”可以通过链式延拓完成。
	\end{definition}
	
	\begin{proof}
		(说明)此条是对“局部幂级数可拼接”的表述,由恒等定理保证在相交的重叠部分粘接一致,因此可得到一致的延拓。
	\end{proof}
	
	\begin{definition}[沿路径的解析延拓与 Monodromy]
		设 \(f\) 在以基点 \(z_0\) 为中心的一个邻域 \(U_0\) 上解析。给定一条从 \(z_0\) 到 \(z_1\) 的路径 \(\gamma\subset\Omega\)(\(\Omega\) 是连通开集),若可以沿着 \(\gamma\) 逐步在小圆盘上继续幂级数,从而在终点 \(z_1\) 得到一个值,这个过程称为“沿路径\(\gamma\)的解析延拓”。若沿不同同伦类的路径延拓得到不同的值,就产生单值性的障碍;Monodromy(单值性/复回绕性)研究沿闭路延拓回到起点时函数值的变化。
	\end{definition}
	
	\begin{definition}[Monodromy 定理(简洁表述)]
		设 \(U\) 为连通开集,\(f\) 在包含基点 \(z_0\) 的某小邻域解析。若对任意闭路 \(\gamma\) (以基点为端点)沿 \(\gamma\) 的解析延拓得到的函数回到起点时与原始函数一致(即单值),则沿路径的延拓对路径同伦类不依赖;反之若依赖则出现非平凡 monodromy。
	\end{definition}
	
	\begin{example}
		经典现象:对函数 \(f(z)=\log z\)(取主支在 \(\arg z\in(0,2\pi)\)),沿绕原点一圈的路径延拓会使 \(\log z\) 增加 \(2\pi i\),产生非平凡 monodromy;为了得到单值需要在复平面上切开一条分支切口,或考虑 \(\log\) 的 Riemann 面。
	\end{example}
	
	\subsection*{解析延拓的常用方法(分类与说明)}
	下面把“如何做延拓”分成若干常见策略,并给出每种方法的核心要点与例子。
	
	\begin{definition}[方法 A:幂级数拼接(直接用 Taylor / Laurent)]
		当你有 \(f\) 在 \(D(a,r)\) 的幂级数表示,并且该幂级数在边界上没有天然障碍时,可以在某边界点 \(b\) 取 \(D(b,\rho)\) 的新的幂级数(由原幂级数的和定义),从而把定义域“向外推”。
	\end{definition}
	
	\begin{example}[幂级数延拓示例]
		函数 \(\sum_{n=0}^\infty z^{n!}\) 在 \(|z|<1\) 内解析,但其收敛圈可能有自然边界;与之相对,几何级数 \(\sum z^n\) 在 \(|z|<1\) 的和 \(1/(1-z)\) 可延拓到 \(\mathbb C\setminus\{1\}\)。
	\end{example}
	
	\begin{definition}[方法 B:通过积分表示延拓(Cauchy 积分、Mellin/Bromwich 变换等)]
		若 \(f\) 有某种积分表示(例如 Mellin 变换、Laplace 逆变换、Cauchy 积分式、Euler 型积分等),并且该积分在较大区域里依然有意义或可解析延拓,则该表示直接给出了延拓。例如 Gamma 函数:
		\[
		\Gamma(s)=\int_0^\infty t^{s-1}e^{-t}\,dt \quad(\Re s>0)
		\]
		通过分部积分得到函数方程 \(\Gamma(s+1)=s\Gamma(s)\),利用递推可把定义延拓到除非正整数外的整个复平面。
	\end{definition}
	
	\begin{example}[Gamma 函数的延拓]
		使用递推关系并用极点解析可以把 \(\Gamma(s)\) 延拓为亚纯函数(meromorphic),其在非正整数处有极点。此处的关键不是直接把积分交换顺序,而是利用函数方程与插值来延拓。
	\end{example}
	
	\begin{definition}[方法 C:利用函数方程或代数关系(functional equations)]
		许多重要函数(比如 Riemann zeta)满足函数方程或可与更好行为的函数组合。例如
		\[
		\eta(s)=\sum_{n=1}^\infty\frac{(-1)^{n-1}}{n^s}
		\]
		(Dirichlet eta)在 \(\Re s>0\) 绝对收敛且与 \(\zeta\) 通过 \(\zeta(s)=(1-2^{1-s})^{-1}\eta(s)\) 相连,从而给出 \(\zeta\) 的延拓(除了 \(s=1\) 的极点)。
	\end{definition}
	
	\begin{example}[Riemann zeta 的经典延拓思路]
		用 Mellin 变换结合 Jacobi theta 函数的变换性质可推导出 zeta 的函数方程,从而把 \(\zeta(s)\) 延拓到整个复平面(仅在 \(s=1\) 有极点)。
	\end{example}
	
	\begin{definition}[方法 D:用微分方程(解析延拓 via ODE)]
		若 \(f\) 是某线性常微分方程(系数为解析函数)的解,则在连通域上可解析延拓该解,障碍仅来自系数的奇点(Fuchsian 理论、等)。因此微分方程是构造全纯延拓的强大工具(例如超几何函数的延拓)。
	\end{definition}
	
	\begin{example}[常见:Gauss 超几何函数]
		Gauss 超几何函数 \({}_2F_1(a,b;c;z)\) 由幂级数在 \(|z|<1\) 定义,但通过解析延拓可推广到 \( \mathbb C\setminus[1,\infty)\)(沿不同路径可得到不同分支)。
	\end{example}
	
	\subsection*{延拓中的障碍与自然边界}
	
	\begin{definition}[自然边界(natural boundary)]
		若函数的定义域主连通延拓到某界面仍无法通过任何方法继续(即任意穿过该界点的延拓都会遇到本性奇点的堆积),则称该界面为自然边界。通俗地说:没有任何更大的域可以包含原函数的解析延拓。
	\end{definition}
	
	\begin{example}
		函数 \(\sum_{n=0}^\infty z^{n!}\) 的收敛区 \(|z|<1\) 的边界单位圆是自然边界(这是经典构造),因单位圆上有密集奇性使得无法跨越。
	\end{example}
	
	\subsection*{Riemann 面与全局单值化}
	为了解决多值延拓(比如 \(\sqrt{z},\log z\))的问题,引入 Riemann 面的观点:把每个分支当作一层,把这些层沿分支点粘接起来,得到一个连通的曲面(Riemann 面),此后函数在该面上成为单值的全纯函数。Riemann 面给出了将多值函数“升格”为单值全纯函数的最自然几何框架。
	
	\begin{example}
		\(\sqrt{z}\) 的 Riemann 面由两层复平面沿 \(0\)(和 \(\infty\))粘接而成;\(\log z\) 的 Riemann 面是无限层的螺旋面。
	\end{example}
	
	\subsection*{单值性、Monodromy 群与解析延拓的代数描述}
	沿闭路的延拓作用在函数的分支集合上,构成一个置换群,即 Monodromy 群。若该群平凡,则全局单值,若非平凡则需要 Riemann 面来“展开”单值化。Monodromy 群的研究将解析延拓与代数拓扑联系起来。
	
	\begin{definition}[Monodromy 群(概要)]
		给定基点 \(z_0\) 与函数的一个局部分支,沿 \(\pi_1(\Omega,z_0)\)(基本群)中路的延拓作用给出对局部分支的置换,从而得到 Monodromy 表示 \(\pi_1(\Omega,z_0)\to S_m\)(若分支数有限)。
	\end{definition}
	
	\subsection*{解析延拓的唯一性与恒等定理的角色}
	解析延拓的一个核心性质是唯一性:若两个全纯函数在连通域的一个有聚点的集合上相等,则它们在该连通域上一致。这保证了“沿两条同伦路径延拓若在交集一致则合并”为可行操作。
	
	\begin{definition}[解析延拓的唯一性(恒等定理的应用)]
		若 \(f\) 在 \(U\) 全纯,\(g\) 在 \(V\) 全纯,且 \(U\cap V\) 含有聚点并且 \(f=g\) 在 \(U\cap V\),则存在并且唯一的全纯函数 \(h\) 在 \(U\cup V\) 上延拓 \(f\) 与 \(g\)。
	\end{definition}
	
	\subsection*{若干重要实例(步骤式)}
	
	\begin{example}[例:用幂级数拼接把 \(1/(1-z)\) 从 \(|z|<1\) 延拓到 \(\mathbb C\setminus\{1\}\)]
		在 \(|z|<1\) 中有 \(\sum z^n = 1/(1-z)\)。但 \(1/(1-z)\) 作为有理函数在整个复平面解析延拓,除了 \(z=1\) 的极点外无其他奇点;这说明幂级数的和已是一个可以在更大域解释的函数(代数延拓非常直接)。
	\end{example}
	
	\begin{example}[例:Gamma 函数通过函数方程延拓]
		起始定义 \(\Gamma(s)=\int_0^\infty t^{s-1}e^{-t}\,dt\) 仅对 \(\Re s>0\) 收敛。利用 \(\Gamma(s+1)=s\Gamma(s)\) 递推以及对负整数点的极点结构可把 \(\Gamma\) 延拓为在全平面上亚纯(meromorphic)的函数(非正整数处为极点)。这是“利用函数方程与解析延拓”的经典范例。
	\end{example}
	
	\begin{example}[例:Riemann zeta 函数的延拓(概要)]
		从 Dirichlet eta 函数 \(\eta(s)=\sum_{n\ge1}(-1)^{n-1}n^{-s}\)(对 \(\Re s>0\) 绝对/条件收敛)和关系
		\[
		\zeta(s) = \frac{1}{1-2^{1-s}}\eta(s)
		\]
		得到 \(\zeta(s)\) 在 \(\Re s>0\) 的延拓(除 \(s=1\)),再结合 theta 函数或 Mellin 变换得到全平面的解析延拓与函数方程(详见专著)。
	\end{example}
	
	\subsection*{练习与思考题(推荐)}
	\begin{enumerate}
		\item 用分部积分从 Euler 型积分推导出 \(\Gamma(s+1)=s\Gamma(s)\),并据此把 \(\Gamma\) 延拓到 \(\Re s\le 0\) 的域(说明极点位置)。
		\item 对函数 \(\sum_{n=0}^\infty z^{2^n}\),讨论其收敛域与是否存在自然边界。
		\item 证明:若 \(f\) 在连通域 \(\Omega\) 的每条闭路沿解析延拓均返回原值(monodromy 平凡),则 \(f\) 沿任意路径的延拓都与路径同伦类无关。
		\item 展示如何用钥匙孔轮廓把 \(\int_0^\infty \dfrac{x^{\alpha-1}}{1+x}\,dx\)(\(0<\Re\alpha<1\))与 Beta/Gamma 函数联系起来,从而证明公式。
	\end{enumerate}
	

	\chapter{复积分的计算:留数定理}
	\section{工具}
	\begin{definition}[留数]
		解析函数\(f(z)\)的Laurent展开中的\(-1\)次项的系数就是留数.即
		\[f(z)=\sum_{n=-\infty}^\infty a_nz^n\]中,留数是\(a_{-1}\).
	\end{definition}
	\begin{definition}[留数定理] 
		设 \(f\) 在简单闭曲线 \(\Gamma\) 内解析,除有限个孤立极点 \(a_k\)。若 \(\Gamma\) 逆时针,则
		\[
		\oint_{\Gamma} f(z)\,dz = 2\pi i \sum_k \operatorname{Res}(f;a_k).
		\]
	\end{definition}
	
	\begin{theorem}[留数的常用计算公式]
		\begin{enumerate}[leftmargin=*,itemsep=4pt]
			\item 若 \(a\) 是 \(f\) 的简单极点,则
			\[
			\operatorname{Res}(f,a)=\lim_{z\to a}(z-a)f(z).
			\]
			\item 若 \(f(z)=\dfrac{\phi(z)}{(z-a)^m}\) 且 \(\phi\) 在 \(a\) 解析,则
			\[
			\operatorname{Res}(f,a)=\frac{1}{(m-1)!}\phi^{(m-1)}(a).
			\]
			\item 若 \(f=\dfrac{g}{h}\) 且 \(h(a)=0,\ h'(a)\neq0\),则
			\[
			\operatorname{Res}(f,a)=\frac{g(a)}{h'(a)}.
			\]
		\end{enumerate}
	\end{theorem}
	\begin{proof}
		\noindent\textbf{证明法一(用 Laurent 展开):}
		对于每个奇点 \(a_j\),因为是孤立的,所以存在小圆盘 \(D_j=\{z:|z-a_j|<r_j\}\) 使得这些小盘互不相交且都被 \(\Gamma\) 的内部包含,并且 \(f\) 在 \(D_j\setminus\{a_j\}\) 可作 Laurent 展开:
		\[
		f(z)=\sum_{k=-\infty}^{\infty} c_{k}^{(j)}(z-a_j)^k,\qquad 0<|z-a_j|<r_j.
		\]
		在每个小圆周 \(C_j=\{z:|z-a_j|=r_j\}\) 上对该级数逐项积分(在收敛圆环内允许逐项积分)得到
		\[
		\oint_{C_j} f(z)\,dz
		= \sum_{k=-\infty}^{\infty} c_k^{(j)} \oint_{C_j} (z-a_j)^k\,dz.
		\]
		但对整数 \(k\) 有
		\[
		\oint_{C_j} (z-a_j)^k\,dz =
		\begin{cases}
			2\pi i, & k=-1,\\
			0, & k\ne -1.
		\end{cases}
		\]
		因此
		\[
		\oint_{C_j} f(z)\,dz = 2\pi i\, c_{-1}^{(j)} = 2\pi i\,\operatorname{Res}(f,a_j).
		\]
		把 \(\Gamma\) 的积分变形(取 \(\Gamma\) 的内部减去这些小盘边界)可由解析延拓与 Cauchy 定理得到等价:沿 \(\Gamma\) 的积分等于这些小圆周积分的和(方向相反,但注意取向约定),整理符号后便得到
		\[
		\oint_{\Gamma} f(z)\,dz = 2\pi i\sum_{j=1}^n \operatorname{Res}(f,a_j).
		\]
		
		\medskip
		\noindent\textbf{证明法二(利用 Cauchy 原理与区域变形):}
		从拓扑角度,我们可以把 \(\Gamma\) 连续缩形到一组绕每个奇点的正向小圆(注意在缩形过程中避开奇点),由 Cauchy 的定理可知,如果把区域上去掉小盘后 \(f\) 在剩下区域内解析,那么沿边界的积分保持不变。于是积分可以写成这些小圆积分之和(方向约定为正向),每个小圆的积分按上面用 Laurent 展开或用局部表示计算,只留下 \((z-a)^{-1}\) 项的贡献,即 \(2\pi i\) 倍的留数。汇总即得定理结论。
	\end{proof}
	
	\section{复积分计算的一般步骤(Recipe)}
	\begin{enumerate}[leftmargin=*,itemsep=4pt]
		\item 将实积分写为沿实轴的复积分(或把被积函数看作复函数);若含三角函数可写为复指数。
		\item 选择合适闭合轮廓(上/下半圆、钥孔、单位圆等),并判断哪些奇点位于区域内部。
		\item 计算这些奇点的留数。
		\item 检查轮廓上其它弧段的贡献(通常用 Jordan 大弧引理估算),并说明其在 \(R\to\infty\) 时是否消失。
		\item 应用留数定理,按需取实部/虚部或主值 (PV) 得到原实积分。
	\end{enumerate}
	
	\section{Jordan 大小弧引理}
	\subsection{Jordan 大弧引理(常用版本)}
	
	\begin{definition}[Jordan 大弧引理]
		 设 \(k>0\)。对每 \(R>0\) 令 \(C_R\) 为以原点为中心、半径 \(R\) 的上半圆 \(z=R e^{i\theta}\), \(\theta\in[0,\pi]\)。若 \(f\) 在该半盘上解析且在弧上满足
		\[
		M_R:=\max_{\theta\in[0,\pi]}|f(R e^{i\theta})|,
		\]
		则对于被积子 \(e^{ikz}f(z)\) 有估计
		\[
		\left|\int_{C_R} e^{ikz} f(z)\,dz\right|
		\le R M_R \int_0^\pi e^{-kR\sin\theta}\,d\theta
		\le \frac{\pi}{k} M_R.
		\]
		于是若 \(M_R\to0\)(例如 \(f(z)=o(1)\) 或 \(M_R=O(R^{-\alpha})\)),则弧积分趋于 \(0\)。
	\end{definition}
	
	\begin{proof}
	参数化圆弧 \(C_R\) 为 \(z=Re^{i\theta}\) (\(\theta\in[0,\pi]\)),则 \(dz=iR e^{i\theta}d\theta\)。因此
	\[
	\left|\int_{C_R} e^{ikz} f(z)\,dz\right|
	\le \int_0^\pi \big|e^{ikRe^{i\theta}} f(Re^{i\theta})\, iR e^{i\theta}\big|\,d\theta
	\le R M_R \int_0^\pi \big|e^{ikRe^{i\theta}}\big|\,d\theta.
	\]
	注意对 \(k>0\) 有
	\[
	\big|e^{ikRe^{i\theta}}\big| = e^{-kR\sin\theta}.
	\]
	于是
	\[
	\left|\int_{C_R} e^{ikz} f(z)\,dz\right|
	\le R M_R \int_0^\pi e^{-kR\sin\theta}\,d\theta.
	\]
	利用对称性并在 \([0,\pi/2]\) 上用不等式 \(\sin\theta\ge \tfrac{2\theta}{\pi}\)(可直接验证),得
	\[
	\int_0^\pi e^{-kR\sin\theta}\,d\theta
	=2\int_0^{\pi/2} e^{-kR\sin\theta}\,d\theta
	\le 2\int_0^{\pi/2} e^{-kR\cdot \tfrac{2\theta}{\pi}}\,d\theta
	= \frac{\pi}{kR}\big(1-e^{-kR}\big)
	\le \frac{\pi}{kR}.
	\]
	代回上式得到
	\[
	\left|\int_{C_R} e^{ikz} f(z)\,dz\right|
	\le R M_R \cdot \frac{\pi}{kR} = \frac{\pi}{k} M_R.
	\]
	当 \(M_R\to0\) 时右端趋于 \(0\),从而弧积分趋于 \(0\)。对 \(k<0\) 改用下半圆并类似估计。
\end{proof}
	
	\subsection{Jordan 小弧引理(绕过实轴上极点)}
	
	\begin{definition}[小弧引理,简单极点情况]
		设 \(a\in\mathbb R\),且 \(f\) 在 \(a\) 附近(去掉 \(a\))解析,且在 \(a\) 处存在简单极点:
		\[
		f(z)=\frac{r}{z-a}+h(z),\qquad h \text{ 在 } a \text{ 解析}.
		\]
		令 \(C_\varepsilon\) 为以 \(a\) 为中心、半径 \(\varepsilon\) 的小半圆(上半圆,参数 \(\theta\in[0,\pi]\) 从右到左)。则当 \(\varepsilon\to0^+\) 时
		\[
		\int_{C_\varepsilon} f(z)\,dz \to i\pi\,r = i\pi\,\operatorname{Res}(f,a).
		\]
		若绕行下半圆则极限为 \(-i\pi\,\operatorname{Res}(f,a)\)。
	\end{definition}
	\begin{proof}
		将上小半弧参数化为 \(z=a+\varepsilon e^{i\theta}\), \(\theta\in[0,\pi]\)。则
		\[
		\int_{C_\varepsilon} \frac{r}{z-a}\,dz
		= \int_0^\pi \frac{r}{\varepsilon e^{i\theta}} \, i\varepsilon e^{i\theta}\,d\theta
		= i r \int_0^\pi d\theta = i\pi r.
		\]
		而余项
		\(\left|\int_{C_\varepsilon} h(z)\,dz\right|\le \pi\varepsilon \max_{C_\varepsilon}|h|\),
		由 \(h\) 在 \(a\) 解析可知上界随 \(\varepsilon\to0\) 趋于 \(0\)。因此小弧积分的极限为 \(i\pi r\)。
	\end{proof}

	
	\section{典型例题}
	
	\begin{example}
		 计算
		\[
		I_1=\int_{-\infty}^\infty \frac{dx}{x^2+1}.
		\]
		\textbf{要点:} 令 \(f(z)=1/(z^2+1)\)。上半平面包含极点 \(i\)。留数 \(\operatorname{Res}(f,i)=1/(2i)\)。大弧因 \(f=O(1/R^2)\) 而贡献 \(0\)。由留数定理,
		\[
		I_1 = 2\pi i\cdot\frac{1}{2i}=\pi.
		\]
	\end{example}
	
	\begin{example}
		计算
		\[
		I_2=\int_{-\infty}^\infty \frac{\cos x}{x^2+1}\,dx.
		\]
		\textbf{要点:} 写作实部 \(I_2=\Re \int_{-\infty}^\infty \dfrac{e^{ix}}{x^2+1}\,dx\)。取 \(F(z)=e^{iz}/(z^2+1)\) 并闭合于上半平面。留数在 \(i\): \(\operatorname{Res}(F,i)=e^{ii}/(2i)=e^{-1}/(2i)\)。因此整体积分为 \(\pi e^{-1}\),取实部得 \(I_2=\pi e^{-1}\)。
	\end{example}
	
	\begin{example}
		 周期积分
		\[
		I_3=\int_0^{2\pi}\frac{d\theta}{a+b\cos\theta},\qquad |a|>|b|.
		\]
		\textbf{要点:} 用 \(z=e^{i\theta}\) 代换(\(\cos\theta=(z+z^{-1})/2,\ d\theta=dz/(iz)\)),把积分化为单位圆上的有理函数积分,找圆内极点并求留数。结果为
		\[
		I_3=\frac{2\pi}{\sqrt{a^2-b^2}}.
		\]
	\end{example}
	
	\begin{example}
		钥孔轮廓:
		\[
		I_4(\alpha)=\int_0^\infty\frac{x^{\alpha-1}}{1+x}\,dx,\qquad 0<\alpha<1.
		\]
		\textbf{要点:} 考虑 \(f(z)=z^{\alpha-1}/(1+z)\)(主值分支),用钥孔轮廓绕过正实轴,内部包含 \(z=-1\)。计算并整理相位差,得到
		\[
		I_4(\alpha)=\frac{\pi}{\sin(\pi\alpha)}.
		\]
	\end{example}
	
	\begin{example}
		主值积分(傅里叶主值)
		\[
		\operatorname{PV}\int_{-\infty}^\infty \frac{e^{ix}}{x}\,dx.
		\]
		\textbf{要点:} 设 \(f(z)=e^{iz}/z\)。用上半平面闭合并避开原点(上小半弧),大弧贡献为 \(0\)。上小弧贡献为 \(i\pi\operatorname{Res}(f,0)=i\pi\)。整体积分 \(2\pi i\)(内部只有 \(0\)),因此
		\[
		\operatorname{PV}\int_{-\infty}^\infty \frac{e^{ix}}{x}\,dx = i\pi,
		\]
		并得 \(\int_{-\infty}^\infty \frac{\sin x}{x}\,dx=\pi\)。
	\end{example}
	
	
	\begin{example}
		 计算
		\[
		\int_{-\infty}^{\infty}\frac{x^2}{x^4+1}\,dx.
		\]
		\textbf{解:} 被积函数为有理函数 \(f(z)=\dfrac{z^2}{z^4+1}\)。多项式 \(z^4+1=0\) 的根为
		\[
		e^{i\pi/4},\ e^{3i\pi/4},\ e^{5i\pi/4},\ e^{7i\pi/4}.
		\]
		上半平面包含两个根 \(z_1=e^{i\pi/4}\) 与 \(z_2=e^{3i\pi/4}\),且这些都是简单根。若 \(g(z)=z^4+1\),则 \(g'(z)=4z^3\),对于简单极点 \(a\) 有
		\[
		\operatorname{Res}\!\left(\frac{z^2}{z^4+1},a\right)=\frac{a^2}{g'(a)}=\frac{a^2}{4a^3}=\frac{1}{4a}.
		\]
		因此
		\[
		\operatorname{Res}(f,z_1)=\frac{1}{4z_1},\qquad
		\operatorname{Res}(f,z_2)=\frac{1}{4z_2}.
		\]
		计算两者和:
		\[
		\frac{1}{4}\left(\frac{1}{e^{i\pi/4}}+\frac{1}{e^{3i\pi/4}}\right)
		=\frac{1}{4}\left(e^{-i\pi/4}+e^{-3i\pi/4}\right)
		=\frac{1}{4}\left(\frac{1-i}{\sqrt2}+\frac{-1-i}{\sqrt2}\right)
		=-\frac{i}{2\sqrt2}.
		\]
		由留数定理及 Jordan 大弧引理(弧贡献为 0),
		\[
		\int_{-\infty}^{\infty}\frac{x^2}{x^4+1}\,dx = 2\pi i\cdot\left(-\frac{i}{2\sqrt2}\right)=\frac{\pi}{\sqrt2}.
		\]
	\end{example}
	
	\begin{example}
		 计算
		\[
		\int_{-\infty}^{\infty}\frac{dx}{x^4+1}.
		\]
		\textbf{解要点:} 与上例类似,考虑 \(f(z)=1/(z^4+1)\)。上半平面极点仍为 \(e^{i\pi/4},e^{3i\pi/4}\)。对简单极点 \(a\),
		\[
		\operatorname{Res}\!\left(\frac{1}{z^4+1},a\right)=\frac{1}{g'(a)}=\frac{1}{4a^3}.
		\]
		计算并求和可得总留数之和为 \(\dfrac{1}{2\sqrt2}(-i)\)(与上例计算类似),于是总体积分为 \( \pi/\sqrt2\)。(也可利用偶性:\(\int_0^\infty dx/(1+x^4)=\pi/(2\sqrt2)\))
	\end{example}
	\begin{example}[Fresnel]
		记
		\[
		C=\int_0^\infty \cos(x^2)\,dx,\qquad S=\int_0^\infty \sin(x^2)\,dx.
		\]
		则
		\[
		C=S=\frac{\sqrt{\pi}}{2\sqrt{2}}=\sqrt{\frac{\pi}{8}}.
		\]
	\end{example}
	
	\begin{proof}
		考虑复积分
		\[
		I=\int_0^\infty e^{i x^2}\,dx.
		\]
		做旋转变换:令 \(u=e^{i\pi/4}x\)(在复平面将实正半轴逆时针旋转 \(\pi/4\)),注意 \(du=e^{i\pi/4}dx\) 且 \(e^{ix^2}=e^{-u^2}\). 在允许交换路径与被积函数(此处可用高斯积分解析延拓)的前提下
		\[
		I=e^{-i\pi/4}\int_0^{\infty e^{i\pi/4}} e^{-u^2}\,du
		= e^{-i\pi/4}\int_0^{\infty} e^{-u^2}\,du
		= e^{-i\pi/4}\cdot\frac{\sqrt{\pi}}{2}.
		\]
		因此
		\[
		\int_0^\infty e^{ix^2}\,dx = \frac{\sqrt{\pi}}{2} e^{-i\pi/4}
		= \frac{\sqrt{\pi}}{2}\frac{1-i}{\sqrt{2}} = \frac{\sqrt{\pi}}{2\sqrt{2}}(1-i).
		\]
		取实部与虚部得
		\[
		C=\Re I=\frac{\sqrt{\pi}}{2\sqrt{2}},\qquad
		S=\Im I=-\frac{\sqrt{\pi}}{2\sqrt{2}}\cdot(-1)=\frac{\sqrt{\pi}}{2\sqrt{2}}.
		\]
		(即 \(C=S=\sqrt{\pi/(8)}\)。)
	\end{proof}
	
	\bigskip
	
	\begin{example}
		证明
		\[
		\int_{-\infty}^{\infty}\frac{dx}{(1+x^2)^{\,n+1}}
		=\sqrt{\pi}\,\frac{\Gamma\!\big(n+\tfrac12\big)}{\Gamma(n+1)},\qquad n>-1/2.
		\]
	\end{example}
	
	\begin{proof}
		由偶性
		\[
		\int_{-\infty}^{\infty}\frac{dx}{(1+x^2)^{n+1}}
		=2\int_0^\infty\frac{dx}{(1+x^2)^{n+1}}.
		\]
		作代换 \(x=\tan\theta\)(\(\theta\in[0,\pi/2)\)),有 \(dx=\sec^2\theta\,d\theta\) 与 \(1+x^2=\sec^2\theta\)。于是
		\[
		\int_0^\infty\frac{dx}{(1+x^2)^{n+1}}
		=\int_0^{\pi/2}\frac{\sec^2\theta\,d\theta}{\sec^{2n+2}\theta}
		=\int_0^{\pi/2}\cos^{2n}\theta\,d\theta
		=\tfrac12 B\!\Big(\tfrac{1}{2},\,n+\tfrac12\Big),
		\]
		其中 \(B\) 为 Beta 函数。于是原积分为
		\[
		2\cdot \tfrac12 B\!\Big(\tfrac{1}{2},\,n+\tfrac12\Big)
		= B\!\Big(\tfrac{1}{2},\,n+\tfrac12\Big)
		= \frac{\Gamma(\tfrac12)\Gamma(n+\tfrac12)}{\Gamma(n+1)}.
		\]
		使用 \(\Gamma(\tfrac12)=\sqrt{\pi}\) 得所需结果。
	\end{proof}
	
	\bigskip
	
	\begin{example}
		当 \(a>1\) 时,求
		\[
		\int_0^{2\pi}\frac{dx}{(a+\cos x)^2}.
		\]
	\end{example}
	
	\begin{proof}
		先令
		\[
		J(a)=\int_0^{2\pi}\frac{dx}{a+\cos x},\qquad a>1.
		\]
		我们已知(或可由变换 \(z=e^{ix}\) 求得)
		\[
		J(a)=\frac{2\pi}{\sqrt{a^2-1}}.
		\]
		注意
		\[
		\frac{d}{da}\frac{1}{a+\cos x} = -\frac{1}{(a+\cos x)^2},
		\]
		所以令
		\[
		K(a)=\int_0^{2\pi}\frac{dx}{(a+\cos x)^2} = -J'(a).
		\]
		直接求导:
		\[
		J'(a)=\frac{d}{da}\Big(2\pi (a^2-1)^{-1/2}\Big)
		=2\pi\cdot\Big(-\tfrac12\Big)(a^2-1)^{-3/2}\cdot 2a
		=-\frac{2\pi a}{(a^2-1)^{3/2}}.
		\]
		因此
		\[
		K(a)=-J'(a)=\frac{2\pi a}{(a^2-1)^{3/2}}.
		\]
	\end{proof}
	
	\bigskip
	
	\begin{example}
		计算
		\[
		\int_0^1 \log(\sin\pi x)\,dx.
		\]
	\end{example}
	
	\begin{proof}
		作变换 \(t=\pi x\):
		\[
		\int_0^1 \log(\sin\pi x)\,dx
		=\frac{1}{\pi}\int_0^\pi \log(\sin t)\,dt.
		\]
		已知(可用对称性与 Fourier 展开或复方法得到)
		\[
		\int_0^\pi \log(\sin t)\,dt = -\pi\log 2.
		\]
		因此原积分为 \(-\log 2\).
	\end{proof}
	
	\bigskip
	
	\begin{example}
		对 \(a>0\),证明
		\[
		\int_0^\infty \frac{\log x}{x^2+a^2}\,dx = \frac{\pi}{2a}\log a.
		\]
	\end{example}
	
	\begin{proof}
		作换元 \(x=a t\) 得
		\[
		\int_0^\infty \frac{\log x}{x^2+a^2}\,dx
		= \int_0^\infty \frac{\log(a t)}{a^2(t^2+1)} a\,dt
		= \frac{1}{a}\int_0^\infty \frac{\log a + \log t}{t^2+1}\,dt.
		\]
		拆开:
		\[
		\frac{1}{a}\log a \int_0^\infty\frac{dt}{1+t^2}
		+\frac{1}{a}\int_0^\infty\frac{\log t}{1+t^2}\,dt.
		\]
		第一个积分等于 \(\frac{\pi}{2}\)。第二个积分记为 \(I_0\)。对 \(I_0\) 做 \(t\mapsto 1/t\):
		\[
		I_0=\int_0^\infty\frac{\log t}{1+t^2}\,dt
		=\int_\infty^0\frac{\log(1/u)}{1+1/u^2}\cdot\frac{-du}{u^2}
		=\int_0^\infty\frac{-\log u}{1+u^2}\,du = -I_0.
		\]
		因此 \(I_0=0\)。最终结果为
		\[
		\frac{1}{a}\log a\cdot\frac{\pi}{2} = \frac{\pi}{2a}\log a.
		\]
	\end{proof}
	
	
	\section{常用技巧与陷阱}
	\begin{itemize}[leftmargin=*,itemsep=4pt]
		\item 选择闭合半平面时必须与 \(e^{ikz}\) 中 \(k\) 的符号匹配:若 \(k>0\) 用上半平面,若 \(k<0\) 用下半平面。
		\item 若极点位于实轴,需用主值并加上小弧贡献(Jordan 小弧引理)。
		\item 对有理函数 \(P/Q\),若 \(\deg Q - \deg P\ge2\),则在大弧上被积函数通常衰减得足够快(弧贡献 \(0\))。
		\item 注意路径方向(顺时针/逆时针)会影响符号(\(2\pi i\) 的正负)。
	\end{itemize}
	
	\section{快速参考公式}
	\begin{align*}
		&\int_{-\infty}^\infty \frac{dx}{x^2+a^2}=\frac{\pi}{a},\qquad a>0,\\
		&\int_{-\infty}^\infty \frac{\cos(bx)}{x^2+a^2}\,dx=\frac{\pi}{a}e^{-a|b|},\\
		&\int_0^\infty \frac{x^{\alpha-1}}{1+x}\,dx=\frac{\pi}{\sin(\pi\alpha)},\quad 0<\alpha<1.
	\end{align*}

\end{document}

